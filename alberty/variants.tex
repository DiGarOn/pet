\documentclass[a4paper,14pt]{extarticle}
\usepackage[utf8]{inputenc}
\usepackage[T2A]{fontenc}
\usepackage{titlesec}
\usepackage{tabu}
\usepackage{booktabs}% for better rules in the table
\usepackage{lipsum}
\usepackage{amsmath}
\usepackage{amsthm}
\usepackage{lscape}
\usepackage{amsfonts}
\usepackage{mathrsfs}
\usepackage{multicol}
\usepackage[russian]{babel}
\usepackage{graphicx}
\usepackage[landscape]{geometry}
\usepackage{pgfplots}
\usepackage{float}
\geometry{left=2.5cm}
\geometry{right=2.5cm}
\geometry{top=2.5cm}
\geometry{bottom=2.5cm}
\pgfplotsset{compat=1.9}
\usepackage{alphalph}

%
%
%\parindent=0.75cm
%
\frenchspacing
%\sloppy
\hyphenpenalty=50

\usepackage{array}
\usepackage{makecell}
\newcolumntype{x}[1]{>{\centering\arraybackslash}p{#1}}

\usepackage{tikz}
\usetikzlibrary{calc}
\usepackage{zref-savepos}

\newcounter{DiagonalizedEntry}
\renewcommand*{\theDiagonalizedEntry}{NTE-\the\value{DiagonalizedEntry}}

\newcommand*{\diagonalize}[2]{%
  \multicolumn{1}{@{}c@{}|}{%
    \stepcounter{DiagonalizedEntry}%
    \vadjust pre{\zsavepos{\theDiagonalizedEntry t}}% top
    \vadjust{\zsavepos{\theDiagonalizedEntry b}}% bottom
    \zsavepos{\theDiagonalizedEntry l}% left
    \hspace{0pt plus 1filll}%
    \zsavepos{\theDiagonalizedEntry r}% right
    \tikz[overlay]{%
      \draw[black]
        let
          \n{llx}={\zposx{\theDiagonalizedEntry l}sp-\zposx{\theDiagonalizedEntry r}sp}, % x left
          \n{urx}={0}, % x right
          \n{lly}={\zposy{\theDiagonalizedEntry b}sp-\zposy{\theDiagonalizedEntry r}sp}, % y bottom
          \n{ury}={\zposy{\theDiagonalizedEntry t}sp-\zposy{\theDiagonalizedEntry r}sp} %  y top
    in
        (\n{llx}, \n{ury}) -- (\n{urx}, \n{lly})
    node[anchor=south west] at (\n{llx}, \n{lly}) {#1}
    node[anchor=north east] at (\n{urx}, \n{ury}) {#2}
    ;
    }% 
  }%
}

%
%---------------------------------------------------------------
\newtheorem{theorem}{Теорема}
\newtheorem{definition}[theorem]{Определение}
\newtheorem{remark}[theorem]{Замечание}
\newtheorem{lemma}[theorem]{Лемма}
\newtheorem{utv}[theorem]{Утверждение}
\newtheorem{proposition}[theorem]{Предложение}
\newtheorem{example}[theorem]{Пример}
\newtheorem{corollary}[theorem]{Следствие}
\newtheorem{prop}[theorem]{Утверждение}
%

\newcommand*{\expect}{\mathsf{M}}
\newcommand*{\prob}{\mathsf{P}}
\newcommand{\eps}{\varepsilon}

%\renewcommand{\tg}{\mbox{tg}\,}
%\renewcommand{\ctg}{\mbox{ctg}\,}
%\renewcommand{\arctg}{\mbox{arctg}\,}
%\renewcommand{\arcctg}{\mbox{arcctg}\,}
%\renewcommand{\arcsin}{\mbox{arcsin}\,}
%\renewcommand{\arccos}{\mbox{arccos}\,}
%\renewcommand{\sh}{\mbox{sh}\,}
%\renewcommand{\ch}{\mbox{ch}\,}
\title{Семинар по математическому анализу} 
\author{А.Б. Чухно} 
\newcommand{\thedate}{\today}
\date{\thedate} 

\newcounter{exercise}[section]
\newenvironment{exercise}[1][]{\refstepcounter{exercise}\par\medskip
%\noindent\makebox[\linewidth]{\rule{\textwidth}{2.25pt}}+
   \noindent\textbf{Вариант~\theexercise. #1}\\
   \noindent\makebox[\linewidth]{\rule{\textwidth}{1.25pt}}
   }
{\vspace{-2.5px}\mbox{}\newline \noindent\makebox[\linewidth]{\rule{\textwidth}{.5pt}}
}

\usepackage{verbatim}

\newenvironment{solution}
{\begin{proof}[\textbf{\textit{Решение}}]}
  {\end{proof}}
%\let\solution\comment
%\let\endsolution\endcomment
\def\ansrel{\mathrel{\stackrel{\mbox{почему?}}{=} }}

\begin{document}
\pagestyle{empty}


\begin{center}
    \textbf{Диск Альберти}
\end{center}
\begin{tikzpicture}


\node[circle,thick,draw=black,fill=white,minimum size=14cm]() at (-2,-8) {};

\node[circle,thick,draw=black,fill=white,minimum size=10cm]() at (11,-8) {};
\foreach \j in {1,...,6}
{
    \pgfmathsetmacro{\jx}{-2+7*cos(11.25*\j) };
    \pgfmathsetmacro{\jy}{-8+7*sin(11.25*\j) };
    \pgfmathsetmacro{\jxx}{-2+6*cos(11.25*\j -5.625) };
    \pgfmathsetmacro{\jyy}{-8+6*sin(11.25*\j-5.625) };
    \pgfmathtruncatemacro{\jsy}{\j+223};
    
    \draw[-,thick](-2, -8) -- (\jx,\jy);
    \node[]() at (\jxx,\jyy) {\symbol{\jsy}};
}
\foreach \j in {7,...,32}
{
    \pgfmathsetmacro{\jx}{-2+7*cos(11.25*\j) };
    \pgfmathsetmacro{\jy}{-8+7*sin(11.25*\j) };
    \pgfmathsetmacro{\jxx}{-2+6*cos(11.25*\j -5.625) };
    \pgfmathsetmacro{\jyy}{-8+6*sin(11.25*\j-5.625) };
    \pgfmathtruncatemacro{\jsy}{\j+223};
    
    \draw[-,thick](-2, -8) -- (\jx,\jy);
    \node[]() at (\jxx,\jyy) {\symbol{\jsy}};
}



\foreach \j in {1,...,6}
{
    \pgfmathsetmacro{\jx}{11+5*cos(11.25*\j) };
    \pgfmathsetmacro{\jy}{-8+5*sin(11.25*\j) };
    
    
    \draw[-,thick](11, -8) -- (\jx,\jy);
    
}
\foreach \j in {7,...,33}
{
    \pgfmathsetmacro{\jx}{11+5*cos(11.25*\j) };
    \pgfmathsetmacro{\jy}{-8+5*sin(11.25*\j) };
    
    
    \draw[-,thick](11, -8) -- (\jx,\jy);
    
}
\end{tikzpicture}

\newpage

\textbf{\textit{Ключ к расшифрованияю:} К\,\, Ф \,\,Э\,\, Н\,\, И\,\, Ь\,\, С\,\, Я\,\, А\,\, Т\,\, У\,\, Б\,\, Ы\,\, Л\,\, Р\,\, В\,\, Щ\,\, Г\,\, Ъ\,\, Д\,\, Е\,\, М\,\, Й\,\, Ц\,\, П\,\, Ч\,\, Ж\,\, Ю\,\, Ш\,\, З\,\, О\,\, Х\,\,}\begin{exercise}\begin{table}[H]
	\centering
	\begin{tabular}{r l}\textbf{Режим 1}  & ХТПЯЩЦТДУАДИЩРЮ ЗАКАЦЯЯЕГБЙЧ ХШПАТ ЪЙШМЙГИЧУВЛЫЫС \\ 
\textbf{Режим 2}  & ЦЧТДЗ ЩЩИЭРРЮШИЦМВЖЗ ИКЪОП ХЮРЧЕШЦЕЕЙЙХ \\ 
\textbf{Режим 3}  & Индикаторная буква: Р \\ 
& ШОХИВЖЖТГИЭШУЙ ГКЫЛ ЩТУЬКЯИЦЕКЖЙ КВХШШЧЮЮПЯДУГБЯ \\ 
\textbf{Режим 4}  & ЮнныутАдивйуМмэзжзЩзиокшЩщсппщ \\ 
\textbf{Режим 5}  & Пароль: ВОЗНИКАЮЩИХ \\ 
& ФОЯТТ ГЩЫЭТБУ РЭСЦБЩБОСХШСШ ЬЪЯМНЖИЗАСЪ \\ 
	\end{tabular} 
\end{table}

\end{exercise}
\begin{solution}
\begin{table}[H]
	\centering
	\begin{tabular}{r l}\textbf{Режим 1}  & ХАРАКТЕРИЗУЕТСЯ ВОЗРАСТАЮЩЕЙ РОЛЬЮ ИНФОРМАЦИОННОЙ \\ 
\textbf{Режим 2}  & СФЕРЫ ПРЕДСТАВЛЯЮЩЕЙ СОБОЙ СОВОКУПНОСТЬ \\ 
\textbf{Режим 2}  & Индикаторная буква: Р \\ 
& ОСУЩЕСТВЛЯЮЩИХ СБОР ФОРМИРОВАНИЕ РАСПРОСТРАНЕНИЕ \\ 
\textbf{Режим 4}  & И ИСПОЛЬЗОВАНИЕ ИНФОРМАЦИИ А \\ 
\textbf{Режим 5}  & Пароль: ВОЗНИКАЮЩИХ \\ 
& ТАКЖЕ СИСТЕМЫ РЕГУЛИРОВАНИЯ ВОЗНИКАЮЩИХ \\ 
	\end{tabular} 
\end{table}

\end{solution}
\begin{exercise}\begin{table}[H]
	\centering
	\begin{tabular}{r l}\textbf{Режим 1}  & ППВ НЦЪВ ЩЭЧИВВОКЯСДЫ ИЯФФУЙЗЦЦ. \\ 
\textbf{Режим 2}  & ЫЩЙВЩБОДИТЯСПЙ ЯТЧНЪ ЩЪЖЛЫЗЯ ЮГЧЧБГЪГНГКСГЮМНС \\ 
\textbf{Режим 3}  & Индикаторная буква: С \\ 
& ЖЬРМЩГРБ ИЬНАЭ ЯЧЩЮССМЪ ГЖФЙВЧЧ \\ 
\textbf{Режим 4}  & УымътЩмэзщРхшхкСеьдрСбмегУъшъфНаелгЙх \\ 
\textbf{Режим 5}  & Пароль: ОБОРОННОЙ \\ 
& ЮМЫЮЫВБМЮВМЫЦ ЫИГРИБПЫЦ С ЯДГЧСК \\ 
	\end{tabular} 
\end{table}

\end{exercise}
\begin{solution}
\begin{table}[H]
	\centering
	\begin{tabular}{r l}\textbf{Режим 1}  & ПРИ ЭТОМ ОБЩЕСТВЕННЫХ ОТНОШЕНИЙ. \\ 
\textbf{Режим 2}  & ИНФОРМАЦИОННАЯ СФЕРА ЯВЛЯЯСЬ СИСТЕМООБРАЗУЮЩИМ \\ 
\textbf{Режим 2}  & Индикаторная буква: С \\ 
& ФАКТОРОМ ЖИЗНИ ОБЩЕСТВА АКТИВНО \\ 
\textbf{Режим 4}  & ВЛИЯЕТ НА СОСТОЯНИЕ ПОЛИТИЧЕСКОЙ \\ 
\textbf{Режим 5}  & Пароль: ОБОРОННОЙ \\ 
& ЭКОНОМИЧЕСКОЙ ОБОРОННОЙ И ДРУГИХ \\ 
	\end{tabular} 
\end{table}

\end{solution}
\begin{exercise}\begin{table}[H]
	\centering
	\begin{tabular}{r l}\textbf{Режим 1}  & СЭИИГЪЖЛБЬДХ УЛРЕЕИЕВВГГС ЛУЫБФФАКНЗ ЯЦММХРНЕД. \\ 
\textbf{Режим 2}  & ЩЭЦАЛБЗЯЦЯЧЦ ПЮШЭЭЪЭЗЗХХА ЮПЖЧВВЙРГБ МЬННЛШГЭФ \\ 
\textbf{Режим 3}  & Индикаторная буква: К \\ 
& ГЪЦЭЛЛШОЬИГФ ЭЕЭГЦЗЖ ЕЫЛЪЖЩЧ ЕЦ \\ 
\textbf{Режим 4}  & ГхъйэкЖхгхяэСьрджеЙсэйыоФвррвуДепжфэЩщфзофЕсшш \\ 
\textbf{Режим 5}  & Пароль: ПРОГРЕССА \\ 
& Ж НЫМД СБЮМКЬНЭПЭТЕ ЦАШФФДЬЦС \\ 
	\end{tabular} 
\end{table}

\end{exercise}
\begin{solution}
\begin{table}[H]
	\centering
	\begin{tabular}{r l}\textbf{Режим 1}  & СОСТАВЛЯЮЩИХ БЕЗОПАСНОСТИ РОССИЙСКОЙ ФЕДЕРАЦИИ. \\ 
\textbf{Режим 2}  & НАЦИОНАЛЬНАЯ БЕЗОПАСНОСТЬ РОССИЙСКОЙ ФЕДЕРАЦИИ \\ 
\textbf{Режим 2}  & Индикаторная буква: К \\ 
& СУЩЕСТВЕННЫМ ОБРАЗОМ ЗАВИСИТ ОТ \\ 
\textbf{Режим 4}  & ОБЕСПЕЧЕНИЯ ИНФОРМАЦИОННОЙ БЕЗОПАСНОСТИ И \\ 
\textbf{Режим 5}  & Пароль: ПРОГРЕССА \\ 
& В ХОДЕ ТЕХНИЧЕСКОГО ПРОГРЕССА \\ 
	\end{tabular} 
\end{table}

\end{solution}
\begin{exercise}\begin{table}[H]
	\centering
	\begin{tabular}{r l}\textbf{Режим 1}  & ЭЙН ТФЭЯВЬАТББЕ ПТЧЧЬ ЕХЦХРООБШЬ. \\ 
\textbf{Режим 2}  & ЯЬЖ ЗФЯКФШВИМЖПЦЦГ АБЫЪЪЭЪЛЛВВПЧ УЯТАООЬЗКЖ \\ 
\textbf{Режим 3}  & Индикаторная буква: Г \\ 
& ЬМДДЗЫФЪГ ЦМДЛЩНЯДГО ВБЛЛАПЬХЮ ШЕЫЖУМШЮЮЗЗЪ \\ 
\textbf{Режим 4}  & ОннбоесОыботжбЦъыхитшНдаегьь \\ 
\textbf{Режим 5}  & Пароль: СОВОКУПНОСТЬЮ \\ 
& РБНЫЩЙПЧСЕЙЪТВ ВННЩВ СЪРБЛУСЬШАСМЮЦ ЕЫАЦВРМБЗВЪСЙ \\ 
	\end{tabular} 
\end{table}

\end{exercise}
\begin{solution}
\begin{table}[H]
	\centering
	\begin{tabular}{r l}\textbf{Режим 1}  & ЭТА ЗАВИСИМОСТЬ БУДЕТ ВОЗРАСТАТЬ. \\ 
\textbf{Режим 2}  & ПОД ИНФОРМАЦИОННОЙ БЕЗОПАСНОСТЬЮ РОССИЙСКОЙ \\ 
\textbf{Режим 2}  & Индикаторная буква: Г \\ 
& ФЕДЕРАЦИИ ПОНИМАЕТСЯ СОСТОЯНИЕ ЗАЩИЩЕННОСТИ \\ 
\textbf{Режим 4}  & ЕЕ НАЦИОНАЛЬНЫХ ИНТЕРЕСОВ В \\ 
\textbf{Режим 5}  & Пароль: СОВОКУПНОСТЬЮ \\ 
& ИНФОРМАЦИОННОЙ СФЕРЕ ОПРЕДЕЛЯЮЩИХСЯ СОВОКУПНОСТЬЮ \\ 
	\end{tabular} 
\end{table}

\end{solution}
\begin{exercise}\begin{table}[H]
	\centering
	\begin{tabular}{r l}\textbf{Режим 1}  & СМДЗГЗКЙЗЖЛУЙМЭЮ ЛГМЯГЬЩЫХ ЯНГССААЗ ЭЕБЙФФЩЛ \\ 
\textbf{Режим 2}  & Ы СЩЪДНОЛЛЛШЧ. КИАЖИПНЫ ЮЦЬЖЖШШГ \\ 
\textbf{Режим 3}  & Индикаторная буква: Ф \\ 
& Н АЫЪБЫЯЖВКЬНЭЭШ НЬМХД ЕЫММЯХЯИЦМЭ \\ 
\textbf{Режим 4}  & ТбчртеЙозйыоХятлыщМязяабОсыббчОдлрок \\ 
\textbf{Режим 5}  & Пароль: ГРАЖДАНИНА \\ 
& АДДАМЛЛИ Б БФРРОДМБГН МГ \\ 
	\end{tabular} 
\end{table}

\end{exercise}
\begin{solution}
\begin{table}[H]
	\centering
	\begin{tabular}{r l}\textbf{Режим 1}  & СБАЛАНСИРОВАННЫХ ИНТЕРЕСОВ ЛИЧНОСТИ ОБЩЕСТВА \\ 
\textbf{Режим 2}  & И ГОСУДАРСТВА. ИНТЕРЕСЫ ЛИЧНОСТИ \\ 
\textbf{Режим 2}  & Индикаторная буква: Ф \\ 
& В ИНФОРМАЦИОННОЙ СФЕРЕ ЗАКЛЮЧАЮТСЯ \\ 
\textbf{Режим 4}  & В РЕАЛИЗАЦИИ КОНСТИТУЦИОННЫХ ПРАВ \\ 
\textbf{Режим 5}  & Пароль: ГРАЖДАНИНА \\ 
& ЧЕЛОВЕКА И ГРАЖДАНИНА НА \\ 
	\end{tabular} 
\end{table}

\end{solution}
\begin{exercise}\begin{table}[H]
	\centering
	\begin{tabular}{r l}\textbf{Режим 1}  & ДШОООЧ Д ЩЕЖДЕВЭЦАЯ БЗ \\ 
\textbf{Режим 2}  & ЫЕГВБЮЬБОЮЯФЗ ХНТЭНОГСЦЙ В ЕПШРПЫЦЬЧ \\ 
\textbf{Режим 3}  & Индикаторная буква: Ч \\ 
& ЯТУВЖЬЬЕШМШЙЫ ЧЩ ГТЦЦБОТЩВВТ СОЯУААЬ \\ 
\textbf{Режим 4}  & РъдлкЪцзрхЕэьсшИпыбыЗеэряАйбйыЫхфжрИгъяъЫе \\ 
\textbf{Режим 5}  & Пароль: РАЗВИТИЯ \\ 
& ЙМВЕВЕТГКЮЗЖКЙЪУШ ПЗЙСГЙВЛ А ВВРВТ \\ 
	\end{tabular} 
\end{table}

\end{exercise}
\begin{solution}
\begin{table}[H]
	\centering
	\begin{tabular}{r l}\textbf{Режим 1}  & ДОСТУП К ИНФОРМАЦИИ НА \\ 
\textbf{Режим 2}  & ИСПОЛЬЗОВАНИЕ ИНФОРМАЦИИ В ИНТЕРЕСАХ \\ 
\textbf{Режим 2}  & Индикаторная буква: Ч \\ 
& ОСУЩЕСТВЛЕНИЯ НЕ ЗАПРЕЩЕННОЙ ЗАКОНОМ \\ 
\textbf{Режим 4}  & ДЕЯТЕЛЬНОСТИ ФИЗИЧЕСКОГО ДУХОВНОГО И \\ 
\textbf{Режим 5}  & Пароль: РАЗВИТИЯ \\ 
& ИНТЕЛЛЕКТУАЛЬНОГО РАЗВИТИЯ А ТАКЖЕ \\ 
	\end{tabular} 
\end{table}

\end{solution}
\begin{exercise}\begin{table}[H]
	\centering
	\begin{tabular}{r l}\textbf{Режим 1}  & В ДБНГЖЫ РЪПГЪЛКМСЬ УШХУЯШЛЗЙДЩУМП \\ 
\textbf{Режим 2}  & ПЕЭЦШС АБЫЪЪЭЪЛЛВВП. НЯБЗЯЮСВ ФДУМККВЫ \\ 
\textbf{Режим 3}  & Индикаторная буква: Щ \\ 
& Ъ ЦЮФЖЮЙТОВЕЪГГБ ЪЕИРЭ НЮИИЙРЙДСИГ \\ 
\textbf{Режим 4}  & НьгиаеЙьюлюнПкккьуТрчржцЛвпмижНгемб \\ 
\textbf{Режим 5}  & Пароль: УПРОЧЕНИИ \\ 
& Ы МКЫЭ ЬЦТЕВ БЗРЯБАГЫЪ \\ 
	\end{tabular} 
\end{table}

\end{exercise}
\begin{solution}
\begin{table}[H]
	\centering
	\begin{tabular}{r l}\textbf{Режим 1}  & В ЗАЩИТЕ ИНФОРМАЦИИ ОБЕСПЕЧИВАЮЩЕЙ \\ 
\textbf{Режим 2}  & ЛИЧНУЮ БЕЗОПАСНОСТЬ. ИНТЕРЕСЫ ОБЩЕСТВА \\ 
\textbf{Режим 2}  & Индикаторная буква: Щ \\ 
& В ИНФОРМАЦИОННОЙ СФЕРЕ ЗАКЛЮЧАЮТСЯ \\ 
\textbf{Режим 4}  & В ОБЕСПЕЧЕНИИ ИНТЕРЕСОВ ЛИЧНОСТИ \\ 
\textbf{Режим 5}  & Пароль: УПРОЧЕНИИ \\ 
& В ЭТОЙ СФЕРЕ УПРОЧЕНИИ \\ 
	\end{tabular} 
\end{table}

\end{solution}
\begin{exercise}\begin{table}[H]
	\centering
	\begin{tabular}{r l}\textbf{Режим 1}  & ДДЧЖМЖТЖВР ЙЪУСФЛЯС ЫЫЮШЯЖЬЧН ЬФАЧЩЖУЖЖРП \\ 
\textbf{Режим 2}  & ЕОКФГЫЮЮЮУЯ УДММУЯЬЫСЬ И УАЗШШСЮЕКЖЧ \\ 
\textbf{Режим 3}  & Индикаторная буква: Х \\ 
& ЛХЙФЫЫЮЗИННЦФ НХДЖРОЕУ Ы ЛЮШЕЯГГР \\ 
\textbf{Режим 4}  & ТцуйцберЭвууъщддЧффпфсбоГфйэыехэЖложубышВв \\ 
\textbf{Режим 5}  & Пароль: ЗАКЛЮЧАЮТСЯ \\ 
& Х ЩЛФТТЕЮХРМУМРЙ ЫЛЛБР ЛЯЯДОИЮЦЖЫА \\ 
	\end{tabular} 
\end{table}

\end{exercise}
\begin{solution}
\begin{table}[H]
	\centering
	\begin{tabular}{r l}\textbf{Режим 1}  & ДЕМОКРАТИИ СОЗДАНИИ ПРАВОВОГО СОЦИАЛЬНОГО \\ 
\textbf{Режим 2}  & ГОСУДАРСТВА ДОСТИЖЕНИИ И ПОДДЕРЖАНИИ \\ 
\textbf{Режим 2}  & Индикаторная буква: Х \\ 
& ОБЩЕСТВЕННОГО СОГЛАСИЯ В ДУХОВНОМ \\ 
\textbf{Режим 4}  & ОБНОВЛЕНИИ РОССИИ. ИНТЕРЕСЫ ГОСУДАРСТВА \\ 
\textbf{Режим 5}  & Пароль: ЗАКЛЮЧАЮТСЯ \\ 
& В ИНФОРМАЦИОННОЙ СФЕРЕ ЗАКЛЮЧАЮТСЯ \\ 
	\end{tabular} 
\end{table}

\end{solution}
\begin{exercise}\begin{table}[H]
	\centering
	\begin{tabular}{r l}\textbf{Режим 1}  & В ЗЧГЛАЕРЛ ЦЕЛВЭЯЯ ФЯЖ \\ 
\textbf{Режим 2}  & ЕЩХЖЮЧДФЙЙУЕ МИУИТЪЯЗ ЛУЫБФФАКНЗ ЮХЬОХЖЛЭДПЙММВ \\ 
\textbf{Режим 3}  & Индикаторная буква: Р \\ 
& ЙЖКЮУЖЖЦЧВЙЙЪШ ИУЗ ЛФЮЬФХЦЫЗШ ЗКОФФЙХХФЪЦМЕФЖ \\ 
\textbf{Режим 4}  & ЧатчюфИмсъьъМчыжкнИстри \\ 
\textbf{Режим 5}  & Пароль: ОБЛАСТИ \\ 
& С РЗАЫЛИБДЖА А ЦЬТБОЖР \\ 
	\end{tabular} 
\end{table}

\end{exercise}
\begin{solution}
\begin{table}[H]
	\centering
	\begin{tabular}{r l}\textbf{Режим 1}  & В СОЗДАНИИ УСЛОВИЙ ДЛЯ \\ 
\textbf{Режим 2}  & ГАРМОНИЧНОГО РАЗВИТИЯ РОССИЙСКОЙ ИНФОРМАЦИОННОЙ \\ 
\textbf{Режим 2}  & Индикаторная буква: Р \\ 
& ИНФРАСТРУКТУРЫ ДЛЯ РЕАЛИЗАЦИИ КОНСТИТУЦИОННЫХ \\ 
\textbf{Режим 4}  & ПРАВ И СВОБОД ЧЕЛОВЕКА \\ 
\textbf{Режим 5}  & Пароль: ОБЛАСТИ \\ 
& И ГРАЖДАНИНА В ОБЛАСТИ \\ 
	\end{tabular} 
\end{table}

\end{solution}
\begin{exercise}\begin{table}[H]
	\centering
	\begin{tabular}{r l}\textbf{Режим 1}  & ПЙЪЧШТЩУК АЫЪБЫЯЖВКХ О ИЭОРПОГРЖЕУ \\ 
\textbf{Режим 2}  & ЧГ Е ЯМЖЛЭ ЖЫВЖЦЫЗУГБФ \\ 
\textbf{Режим 3}  & Индикаторная буква: Б \\ 
& ПВЩНАГУЩГДДА ТЫУЛЛЭББЛОНФККЕО ККЗЖУ ЮШУЫЙУГБЕСЪХ \\ 
\textbf{Режим 4}  & НбмаддбмТцчгтеьйФвуцсывъЕсфэьсшиЧяууфф \\ 
\textbf{Режим 5}  & Пароль: СОЦИАЛЬНОЙ \\ 
& МЫЭЫЖМШАВКЕЯ ЕРЙЖДВСЛБВФЪГ М ЙГЕОСТДГЙЙ \\ 
	\end{tabular} 
\end{table}

\end{exercise}
\begin{solution}
\begin{table}[H]
	\centering
	\begin{tabular}{r l}\textbf{Режим 1}  & ПОЛУЧЕНИЯ ИНФОРМАЦИИ И ПОЛЬЗОВАНИЯ \\ 
\textbf{Режим 2}  & ЕЮ В ЦЕЛЯХ ОБЕСПЕЧЕНИЯ \\ 
\textbf{Режим 2}  & Индикаторная буква: Б \\ 
& НЕЗЫБЛЕМОСТИ КОНСТИТУЦИОННОГО СТРОЯ СУВЕРЕНИТЕТА \\ 
\textbf{Режим 4}  & И ТЕРРИТОРИАЛЬНОЙ ЦЕЛОСТНОСТИ РОССИИ \\ 
\textbf{Режим 5}  & Пароль: СОЦИАЛЬНОЙ \\ 
& ПОЛИТИЧЕСКОЙ ЭКОНОМИЧЕСКОЙ И СОЦИАЛЬНОЙ \\ 
	\end{tabular} 
\end{table}

\end{solution}
\begin{exercise}\begin{table}[H]
	\centering
	\begin{tabular}{r l}\textbf{Режим 1}  & ССДДЧЮБЮЮЗЗЪ Б ТЫЛПМРЩНЛЛУ БЗКБАЗРЮЭШЮ \\ 
\textbf{Режим 2}  & ИШЬАСЬЬЯЯШ Ю ФФЩГШШЖЮТЛЪЯ ЙЬБЬУДАЯ \\ 
\textbf{Режим 3}  & Индикаторная буква: У \\ 
& ЖТУЕЕЕЕНИРРЭЫ Ь ОЭЖФИЬПЩЦФЙООЪШ ЧГГРШМЯЙДАЩЩИР \\ 
\textbf{Режим 4}  & ФъвдгеьрЪжтцниеъВршвзкшзЬяуъьюлдЫчыцачяф \\ 
\textbf{Режим 5}  & Пароль: ФЕДЕРАЦИИ \\ 
& ТФЬЧОЛАЪС ГЭИЬЙГАРЪУ АППДПЦЖЫТ Й \\ 
	\end{tabular} 
\end{table}

\end{exercise}
\begin{solution}
\begin{table}[H]
	\centering
	\begin{tabular}{r l}\textbf{Режим 1}  & СТАБИЛЬНОСТИ В БЕЗУСЛОВНОМ ОБЕСПЕЧЕНИИ \\ 
\textbf{Режим 2}  & ЗАКОННОСТИ И ПРАВОПОРЯДКА РАЗВИТИИ \\ 
\textbf{Режим 2}  & Индикаторная буква: У \\ 
& РАВНОПРАВНОГО И ВЗАИМОВЫГОДНОГО МЕЖДУНАРОДНОГО \\ 
\textbf{Режим 4}  & СОТРУДНИЧЕСТВА. НА ОСНОВЕ НАЦИОНАЛЬНЫХ \\ 
\textbf{Режим 5}  & Пароль: ФЕДЕРАЦИИ \\ 
& ИНТЕРЕСОВ РОССИЙСКОЙ ФЕДЕРАЦИИ В \\ 
	\end{tabular} 
\end{table}

\end{solution}
\begin{exercise}\begin{table}[H]
	\centering
	\begin{tabular}{r l}\textbf{Режим 1}  & ИАРЯАНЙБШФХООП ХФГЮВ ЗЙЦЪЛЕЙКДГО ВВЫЮЫОЮХЛЖЬЗЖЙ \\ 
\textbf{Режим 2}  & Ы МЯБДПСЭ НЮОЧВХ ПЭЯЬЭЙЗШДЙ \\ 
\textbf{Режим 3}  & Индикаторная буква: Щ \\ 
& П ГЮЪИПВЪ ЦМГЛЙБЫТ НЛВЩКЮУУУЧЙ \\ 
\textbf{Режим 4}  & УчппбвюЗыэеэуьЧцфслятИщижыъгОбыяхньЕэнеьфэЬйцли \\ 
\textbf{Режим 5}  & Пароль: СОСТАВЛЯЮЩИЕ \\ 
& АЧУРДРБЧЫВ ЮЧПЧЙР ЙКЖМЗЗХЧ ЦЫЦЮАГПСЧТЫЧ \\ 
	\end{tabular} 
\end{table}

\end{exercise}
\begin{solution}
\begin{table}[H]
	\centering
	\begin{tabular}{r l}\textbf{Режим 1}  & ИНФОРМАЦИОННОЙ СФЕРЕ ФОРМИРУЮТСЯ СТРАТЕГИЧЕСКИЕ \\ 
\textbf{Режим 2}  & И ТЕКУЩИЕ ЗАДАЧИ ВНУТРЕННЕЙ \\ 
\textbf{Режим 2}  & Индикаторная буква: Щ \\ 
& И ВНЕШНЕЙ ПОЛИТИКИ ГОСУДАРСТВА \\ 
\textbf{Режим 4}  & ПО ОБЕСПЕЧЕНИЮ ИНФОРМАЦИОННОЙ БЕЗОПАСНОСТИ. \\ 
\textbf{Режим 5}  & Пароль: СОСТАВЛЯЮЩИЕ \\ 
& ВЫДЕЛЯЮТСЯ ЧЕТЫРЕ ОСНОВНЫЕ СОСТАВЛЯЮЩИЕ \\ 
	\end{tabular} 
\end{table}

\end{solution}
\begin{exercise}\begin{table}[H]
	\centering
	\begin{tabular}{r l}\textbf{Режим 1}  & НМУЮКОЩЖУЖЯК ЪЙЖЫЙУМГИ ГРЩВЬЬЫИЯФ БЮЧЧНЪЯПЦ \\ 
\textbf{Режим 2}  & Г ЙЖКЧЖМАЗРДГЩЩУ ГДНЛФ. УХТЖЦМ \\ 
\textbf{Режим 3}  & Индикаторная буква: У \\ 
& ЮЦЧЧСЯВЭКЖКО БЗЪНАСПЭЪЭВА ЖОЭЕОЪЗЧЫ ЧЙПЦЛЛЕЫВТ \\ 
\textbf{Режим 4}  & АшлылпаЗдььхьуЪскэоъаВцзшшзпАчшлпл \\ 
\textbf{Режим 5}  & Пароль: СОБЛЮДЕНИЕ \\ 
& ААЙБЪДЧМ С ЬБХУ ОТЕХФАЧДСЩ \\ 
	\end{tabular} 
\end{table}

\end{exercise}
\begin{solution}
\begin{table}[H]
	\centering
	\begin{tabular}{r l}\textbf{Режим 1}  & НАЦИОНАЛЬНЫХ ИНТЕРЕСОВ РОССИЙСКОЙ ФЕДЕРАЦИИ \\ 
\textbf{Режим 2}  & В ИНФОРМАЦИОННОЙ СФЕРЕ. ПЕРВАЯ \\ 
\textbf{Режим 2}  & Индикаторная буква: У \\ 
& СОСТАВЛЯЮЩАЯ НАЦИОНАЛЬНЫХ ИНТЕРЕСОВ РОССИЙСКОЙ \\ 
\textbf{Режим 4}  & ФЕДЕРАЦИИ В ИНФОРМАЦИОННОЙ СФЕРЕ \\ 
\textbf{Режим 5}  & Пароль: СОБЛЮДЕНИЕ \\ 
& ВКЛЮЧАЕТ В СЕБЯ СОБЛЮДЕНИЕ \\ 
	\end{tabular} 
\end{table}

\end{solution}
\begin{exercise}\begin{table}[H]
	\centering
	\begin{tabular}{r l}\textbf{Режим 1}  & КНФИИЧЭЭИМЖПЦИЗ ЙЙЬС Б ДНРКЫФ \\ 
\textbf{Режим 2}  & ЩЗЭИЦЧЗЪ Ч ДХРДЩБЦГМЯ А \\ 
\textbf{Режим 3}  & Индикаторная буква: Л \\ 
& ЮЛЙУЖЖВ МДВЙЧСЛЯЗ ЬТВАТИЦЫЗШ Ю \\ 
\textbf{Режим 4}  & МинкгшнЮзюанжкХзлкищрАлхлмщлКидмрэлЙичи \\ 
\textbf{Режим 5}  & Пароль: СОХРАНЕНИЕ \\ 
& ЕХЫШУРЧЩЫД ЙЫЩХЩБ ЬГЧИСБИЮЩА Ш \\ 
	\end{tabular} 
\end{table}

\end{exercise}
\begin{solution}
\begin{table}[H]
	\centering
	\begin{tabular}{r l}\textbf{Режим 1}  & КОНСТИТУЦИОННЫХ ПРАВ И СВОБОД \\ 
\textbf{Режим 2}  & ЧЕЛОВЕКА И ГРАЖДАНИНА В \\ 
\textbf{Режим 2}  & Индикаторная буква: Л \\ 
& ОБЛАСТИ ПОЛУЧЕНИЯ ИНФОРМАЦИИ И \\ 
\textbf{Режим 4}  & ПОЛЬЗОВАНИЯ ЕЮ ОБЕСПЕЧЕНИЕ ДУХОВНОГО \\ 
\textbf{Режим 5}  & Пароль: СОХРАНЕНИЕ \\ 
& ОБНОВЛЕНИЯ РОССИИ СОХРАНЕНИЕ И \\ 
	\end{tabular} 
\end{table}

\end{solution}
\begin{exercise}\begin{table}[H]
	\centering
	\begin{tabular}{r l}\textbf{Режим 1}  & УКЬЧЭЗЙЗЦД ЖШБЫЖЖТБЪГОЦ ЦЬЫББЛЛХЭ СПВЖЬЬЕГ \\ 
\textbf{Режим 2}  & ЮПЯУЛЖББ ЩКГРЬУЛЭКИЦ О ПЯХЪОПЙЖЛ \\ 
\textbf{Режим 3}  & Индикаторная буква: С \\ 
& ЩПВХЕЕЩЫЫКУ Н ЯЧБРННЦФ ФХЭЕЮНЕБМТ \\ 
\textbf{Режим 4}  & ГэнфгоЧепзипИаъмйыОинбсзМъянпн \\ 
\textbf{Режим 5}  & Пароль: ИСПОЛЬЗОВАНИЯ \\ 
& ЙЙЗХКУВВР ЦГСЭМРУЖ УЧГНЧЖБСЕЪЦУЖ МЙЫЫЖИУЪТИДКЗ \\ 
	\end{tabular} 
\end{table}

\end{exercise}
\begin{solution}
\begin{table}[H]
	\centering
	\begin{tabular}{r l}\textbf{Режим 1}  & УКРЕПЛЕНИЕ НРАВСТВЕННЫХ ЦЕННОСТЕЙ ОБЩЕСТВА \\ 
\textbf{Режим 2}  & ТРАДИЦИЙ ПАТРИОТИЗМА И ГУМАНИЗМА \\ 
\textbf{Режим 2}  & Индикаторная буква: С \\ 
& КУЛЬТУРНОГО И НАУЧНОГО ПОТЕНЦИАЛА \\ 
\textbf{Режим 4}  & СТРАНЫ. ДЛЯ ДОСТИЖЕНИЯ ЭТОГО \\ 
\textbf{Режим 5}  & Пароль: ИСПОЛЬЗОВАНИЯ \\ 
& ТРЕБУЕТСЯ ПОВЫСИТЬ ЭФФЕКТИВНОСТЬ ИСПОЛЬЗОВАНИЯ \\ 
	\end{tabular} 
\end{table}

\end{solution}
\begin{exercise}\begin{table}[H]
	\centering
	\begin{tabular}{r l}\textbf{Режим 1}  & ИАРЯАНЙБШФХООП ЙЖКЮУЖЖЦЧВЙЙЪШ Э ЯБВФБХУПЫ \\ 
\textbf{Режим 2}  & ЫОМКББЖШНЭЭЙК ФЩЙЩМОДТ ДЙЕЦДВБСФЙЯС ЛУЫБФФАКНЦФ \\ 
\textbf{Режим 3}  & Индикаторная буква: Й \\ 
& ИЙЛПННЪВ ЪКФЧЫЙЙУЕ ЯЪУЪВСННУИЖ СССЖИЭЕТЖХЗВЧТЧЧЛЦ \\ 
\textbf{Режим 4}  & МэмьнчНндгееВцпкчзДшяпцпЕиеушшОгвыкнЛзьъжоАжуйуаПуе \\ 
\textbf{Режим 5}  & Пароль: СОХРАНЕНИЯ \\ 
& ЦСЩКЛВЯ ЦЪЦЪСВШУНФБН ПЕДВРМАФБН В \\ 
	\end{tabular} 
\end{table}

\end{exercise}
\begin{solution}
\begin{table}[H]
	\centering
	\begin{tabular}{r l}\textbf{Режим 1}  & ИНФОРМАЦИОННОЙ ИНФРАСТРУКТУРЫ В ИНТЕРЕСАХ \\ 
\textbf{Режим 2}  & ОБЩЕСТВЕННОГО РАЗВИТИЯ КОНСОЛИДАЦИИ РОССИЙСКОГО \\ 
\textbf{Режим 2}  & Индикаторная буква: Й \\ 
& ОБЩЕСТВА ДУХОВНОГО ВОЗРОЖДЕНИЯ МНОГОНАЦИОНАЛЬНОГО \\ 
\textbf{Режим 4}  & НАРОДА РОССИЙСКОЙ ФЕДЕРАЦИИ УСОВЕРШЕНСТВОВАТЬ \\ 
\textbf{Режим 5}  & Пароль: СОХРАНЕНИЯ \\ 
& СИСТЕМУ ФОРМИРОВАНИЯ СОХРАНЕНИЯ И \\ 
	\end{tabular} 
\end{table}

\end{solution}
\begin{exercise}\begin{table}[H]
	\centering
	\begin{tabular}{r l}\textbf{Режим 1}  & СЭИИГЪЖЛБЬДХ ЦЧЕЕЯЦ ЩЭЕЦЫЫ-ВФЩЯФВШЯХЭЙК Ч \\ 
\textbf{Режим 2}  & ЦЬСОГЮЮВЧ ЧЦЖЫДЮЫНЛФ ЩЛВРИИБНСК УЖППЭГСЦЙ \\ 
\textbf{Режим 3}  & Индикаторная буква: Е \\ 
& ЭЕЦЭХЕЬМОЯ ЕЦМППЛЙЙПТРЫБПФ УУЧЖЦ О \\ 
\textbf{Режим 4}  & ИмсъьъНяожарМнцжхмЫевшыъВдвшцшЧч \\ 
\textbf{Режим 5}  & Пароль: ПЕРЕДАВАТЬ \\ 
& ТЙШМФЫШЙ ГЙЭЕКЩ ЭЙЖЮКЬУЩ ЗЧНЛДАБЬУЩ \\ 
	\end{tabular} 
\end{table}

\end{exercise}
\begin{solution}
\begin{table}[H]
	\centering
	\begin{tabular}{r l}\textbf{Режим 1}  & СОСТАВЛЯЮЩИХ ОСНОВУ НАУЧНО-ТЕХНИЧЕСКОГО И \\ 
\textbf{Режим 2}  & ДУХОВНОГО ПОТЕНЦИАЛА РОССИЙСКОЙ ФЕДЕРАЦИИ \\ 
\textbf{Режим 2}  & Индикаторная буква: Е \\ 
& ОБЕСПЕЧИТЬ КОНСТИТУЦИОННЫЕ ПРАВА И \\ 
\textbf{Режим 4}  & СВОБОДЫ ЧЕЛОВЕКА И ГРАЖДАНИНА \\ 
\textbf{Режим 5}  & Пароль: ПЕРЕДАВАТЬ \\ 
& СВОБОДНО ИСКАТЬ ПОЛУЧАТЬ ПЕРЕДАВАТЬ \\ 
	\end{tabular} 
\end{table}

\end{solution}
\begin{exercise}\begin{table}[H]
	\centering
	\begin{tabular}{r l}\textbf{Режим 1}  & ППМРЫСГСТЪЮ С ЛШЫТТСААЬЕКГНЫ ЦЮФЖЮЙТОВИ \\ 
\textbf{Режим 2}  & ПУЫЬМ ВЯЩДГЩЗЛ ГРЫРУШАЬ ЯЬФАБДИЛ \\ 
\textbf{Режим 3}  & Индикаторная буква: Ж \\ 
& ОЯТТИЦЧНХИР ЦЮФЖЮЙТОВИ Д МГДДРОБЬИ \\ 
\textbf{Режим 4}  & ЙикстшБбтищеБшющвсЫжлгзюЮкънлцАъймчжЙоатыоЙиннгю \\ 
\textbf{Режим 5}  & Пароль: ЧЕЛОВЕКА \\ 
& АИЛКВ Ш ГУЯМЮЭУ БЬДЯЙЪАВ \\ 
	\end{tabular} 
\end{table}

\end{exercise}
\begin{solution}
\begin{table}[H]
	\centering
	\begin{tabular}{r l}\textbf{Режим 1}  & ПРОИЗВОДИТЬ И РАСПРОСТРАНЯТЬ ИНФОРМАЦИЮ \\ 
\textbf{Режим 2}  & ЛЮБЫМ ЗАКОННЫМ СПОСОБОМ ПОЛУЧАТЬ \\ 
\textbf{Режим 2}  & Индикаторная буква: Ж \\ 
& ДОСТОВЕРНУЮ ИНФОРМАЦИЮ О СОСТОЯНИИ \\ 
\textbf{Режим 4}  & ОКРУЖАЮЩЕЙ СРЕДЫ ОБЕСПЕЧИТЬ КОНСТИТУЦИОННЫЕ \\ 
\textbf{Режим 5}  & Пароль: ЧЕЛОВЕКА \\ 
& ПРАВА И СВОБОДЫ ЧЕЛОВЕКА \\ 
	\end{tabular} 
\end{table}

\end{solution}
\begin{exercise}\begin{table}[H]
	\centering
	\begin{tabular}{r l}\textbf{Режим 1}  & И ОУЧОЮЙЭШКЪ ОЩ ЖЙИЧОА \\ 
\textbf{Режим 2}  & Ы ЕЯЛЬАБЩЖ ЛЖЭЬУ АМЗКЬ \\ 
\textbf{Режим 3}  & Индикаторная буква: С \\ 
& МУЕАГТГТС ВФСХЛСИНЩТ ЭЦФМЪШВЖЮПБ ЦМЗЦЪЬЗЙ \\ 
\textbf{Режим 4}  & ЗвэаэкУжуоцьОдссбчТожццуЦщшихк \\ 
\textbf{Режим 5}  & Пароль: ЧЕЛОВЕКА \\ 
& СЕ ЕОАШЪЮ УЙЮНП БЬЧБШ \\ 
	\end{tabular} 
\end{table}

\end{exercise}
\begin{solution}
\begin{table}[H]
	\centering
	\begin{tabular}{r l}\textbf{Режим 1}  & И ГРАЖДАНИНА НА ЛИЧНУЮ \\ 
\textbf{Режим 2}  & И СЕМЕЙНУЮ ТАЙНУ ТАЙНУ \\ 
\textbf{Режим 2}  & Индикаторная буква: С \\ 
& ПЕРЕПИСКИ ТЕЛЕФОННЫХ ПЕРЕГОВОРОВ ПОЧТОВЫХ \\ 
\textbf{Режим 4}  & ТЕЛЕГРАФНЫХ И ИНЫХ СООБЩЕНИЙ \\ 
\textbf{Режим 5}  & Пароль: ЧЕЛОВЕКА \\ 
& НА ЗАЩИТУ СВОЕЙ ЧЕСТИ \\ 
	\end{tabular} 
\end{table}

\end{solution}
\begin{exercise}\begin{table}[H]
	\centering
	\begin{tabular}{r l}\textbf{Режим 1}  & И ЫШЯЗЖИ ЧЭЕЭХДЗ ЙЧГПЪ \\ 
\textbf{Режим 2}  & ЗДПЫМЛЙХ РСМХБЬНЯМ ААЦПНЙФМХ КЕГКЦДЖЦУЯДЛН \\ 
\textbf{Режим 3}  & Индикаторная буква: Е \\ 
& ЭЬХХАЕЮММ Л ПУДЯЦЦЫ ГЦГКЫЧ \\ 
\textbf{Режим 4}  & ТгйюреерНлмйнрхщВзпкзщкфЫргччжзоСрцелуспЦдуаэьчхСь \\ 
\textbf{Режим 5}  & Пароль: ФЕДЕРАЛЬНЫМ \\ 
& ЬХЪ ЬШТПНЯГЭТД СЬКАЖДЬЦЖРФРТ ЭЛГУДЫКЗФРК \\ 
	\end{tabular} 
\end{table}

\end{exercise}
\begin{solution}
\begin{table}[H]
	\centering
	\begin{tabular}{r l}\textbf{Режим 1}  & И СВОЕГО ДОБРОГО ИМЕНИ \\ 
\textbf{Режим 2}  & УКРЕПИТЬ МЕХАНИЗМЫ ПРАВОВОГО РЕГУЛИРОВАНИЯ \\ 
\textbf{Режим 2}  & Индикаторная буква: Е \\ 
& ОТНОШЕНИЙ В ОБЛАСТИ ОХРАНЫ \\ 
\textbf{Режим 4}  & ИНТЕЛЛЕКТУАЛЬНОЙ СОБСТВЕННОСТИ СОЗДАТЬ УСЛОВИЯ \\ 
\textbf{Режим 5}  & Пароль: ФЕДЕРАЛЬНЫМ \\ 
& ДЛЯ СОБЛЮДЕНИЯ УСТАНОВЛЕННЫХ ФЕДЕРАЛЬНЫМ \\ 
	\end{tabular} 
\end{table}

\end{solution}
\begin{exercise}\begin{table}[H]
	\centering
	\begin{tabular}{r l}\textbf{Режим 1}  & ЗМОФХХЕВКЪЦЯЮЮУМЪ ДЯДЭРАЙНУИИ ЯЧ ШЬЯЯЯЭ \\ 
\textbf{Режим 2}  & У ТЫУЩНООЬЛОМХВХХЧ ЦЮФЖЮЙТОВР АИДЭРЪЯРБОЮЫМ \\ 
\textbf{Режим 3}  & Индикаторная буква: Ц \\ 
& АПНМФЙИ ШВХОЖЛПЪ ЩЕЖДЕВЭЦАЯ С \\ 
\textbf{Режим 4}  & ПхпяазурШфтнвыцтПзшсябтэТтюь \\ 
\textbf{Режим 5}  & Пароль: КОТОРЫЕ \\ 
& ВРЦЛРВЕЛЭШ С РВШЪОХСЫ ЙЭЪЫЧЧД \\ 
	\end{tabular} 
\end{table}

\end{exercise}
\begin{solution}
\begin{table}[H]
	\centering
	\begin{tabular}{r l}\textbf{Режим 1}  & ЗАКОНОДАТЕЛЬСТВОМ ОГРАНИЧЕНИЙ НА ДОСТУП \\ 
\textbf{Режим 2}  & К КОНФИДЕНЦИАЛЬНОЙ ИНФОРМАЦИИ ГАРАНТИРОВАТЬ \\ 
\textbf{Режим 2}  & Индикаторная буква: Ц \\ 
& СВОБОДУ МАССОВОЙ ИНФОРМАЦИИ И \\ 
\textbf{Режим 4}  & ЗАПРЕТ ЦЕНЗУРЫ НЕ ДОПУСКАТЬ \\ 
\textbf{Режим 5}  & Пароль: КОТОРЫЕ \\ 
& ПРОПАГАНДУ И АГИТАЦИЮ КОТОРЫЕ \\ 
	\end{tabular} 
\end{table}

\end{solution}
\begin{exercise}\begin{table}[H]
	\centering
	\begin{tabular}{r l}\textbf{Режим 1}  & СНФНХГККВХТШ ЧАРЫЛЯНВТО ЩЫДИЮЬМЬЬХ ЬЕИКДОП \\ 
\textbf{Режим 2}  & ЩЭЦАЛБЗЯЦЯЯФ ХФЗ ИПЗЧДЦШЕЧЧД ЙЛЕСЯЫЕЕТ \\ 
\textbf{Режим 3}  & Индикаторная буква: Е \\ 
& Ш МЭГЙДТ ЮЛЩЮПЛОРЦК УЭЩЩИЩ \\ 
\textbf{Режим 4}  & ЭвэднщъцШыштфтифЬлйедщфыКрэклаь \\ 
\textbf{Режим 5}  & Пароль: ИНФОРМАЦИИ \\ 
& Ы ДФВЗЬЙАЙЕИЩСБЙЖ ЩИПЪЕВФЕЙЗ Й \\ 
	\end{tabular} 
\end{table}

\end{exercise}
\begin{solution}
\begin{table}[H]
	\centering
	\begin{tabular}{r l}\textbf{Режим 1}  & СПОСОБСТВУЮТ РАЗЖИГАНИЮ СОЦИАЛЬНОЙ РАСОВОЙ \\ 
\textbf{Режим 2}  & НАЦИОНАЛЬНОЙ ИЛИ РЕЛИГИОЗНОЙ НЕНАВИСТИ \\ 
\textbf{Режим 2}  & Индикаторная буква: Е \\ 
& И ВРАЖДЫ ОБЕСПЕЧИТЬ ЗАПРЕТ \\ 
\textbf{Режим 4}  & НА СБОР ХРАНЕНИЕ ИСПОЛЬЗОВАНИЕ \\ 
\textbf{Режим 5}  & Пароль: ИНФОРМАЦИИ \\ 
& И РАСПРОСТРАНЕНИЕ ИНФОРМАЦИИ О \\ 
	\end{tabular} 
\end{table}

\end{solution}
\begin{exercise}\begin{table}[H]
	\centering
	\begin{tabular}{r l}\textbf{Режим 1}  & ЧФГГББЬ ФЭКЬХ ФЗУД ДЙЦ \\ 
\textbf{Режим 2}  & ЧЙК ЭОЪЧЛЗДТ Г ЫЦЧАЪЫ \\ 
\textbf{Режим 3}  & Индикаторная буква: Э \\ 
& УРЕЛРТШГЭФ ШЬЯЯЯЭ Ш ЮОФШЗЖЕ \\ 
\textbf{Режим 4}  & МнпьмЖяэгхСджбуТрчтеПеьдиЪчъшкХылнхНмархСцпаеМфецяБчюбу \\ 
\textbf{Режим 5}  & Пароль: РОССИЙСКОЙ \\ 
& ХЫЦПИПГХШВРЗ ДСЖОЕЛОФУБФЮ ЫНПЬРЮХЫА ЙЪЯЦАЯЯПЫВ \\ 
	\end{tabular} 
\end{table}

\end{exercise}
\begin{solution}
\begin{table}[H]
	\centering
	\begin{tabular}{r l}\textbf{Режим 1}  & ЧАСТНОЙ ЖИЗНИ ЛИЦА БЕЗ \\ 
\textbf{Режим 2}  & ЕГО СОГЛАСИЯ И ДРУГОЙ \\ 
\textbf{Режим 2}  & Индикаторная буква: Э \\ 
& ИНФОРМАЦИИ ДОСТУП К КОТОРОЙ \\ 
\textbf{Режим 4}  & ОГРАНИЧЕН ФЕДЕРАЛЬНЫМ ЗАКОНОДАТЕЛЬСТВОМ. ВТОРАЯ \\ 
\textbf{Режим 5}  & Пароль: РОССИЙСКОЙ \\ 
& СОСТАВЛЯЮЩАЯ НАЦИОНАЛЬНЫХ ИНТЕРЕСОВ РОССИЙСКОЙ \\ 
	\end{tabular} 
\end{table}

\end{solution}
\begin{exercise}\begin{table}[H]
	\centering
	\begin{tabular}{r l}\textbf{Режим 1}  & ФГВВЧАЗРЛ С БВМРВУЗЪНАСЬЬХ САПЭЙ \\ 
\textbf{Режим 2}  & ГППУЭУРЖ Т ЦУЬЭ УРЕЛРТШГЭЯЬИИЖ \\ 
\textbf{Режим 3}  & Индикаторная буква: Л \\ 
& ЮЛЩЮПЛОБЪЫА ЬВГЪЭЗЫЫЫЮЗИННЗ ЭКШПФЙЦЕ ЮПЖЧВВЙРГБ \\ 
\textbf{Режим 4}  & ЮвкхкбЧчвффуБлуъбпЭвщядсЯмтыбыПькзи \\ 
\textbf{Режим 5}  & Пароль: МЕЖДУНАРОДНОЙ \\ 
& ЧЭ УФЮЕЩЦВЗГЯ О ФЧКЦЗЩАОЫЦЩЫХ \\ 
	\end{tabular} 
\end{table}

\end{exercise}
\begin{solution}
\begin{table}[H]
	\centering
	\begin{tabular}{r l}\textbf{Режим 1}  & ФЕДЕРАЦИИ В ИНФОРМАЦИОННОЙ СФЕРЕ \\ 
\textbf{Режим 2}  & ВКЛЮЧАЕТ В СЕБЯ ИНФОРМАЦИОННОЕ \\ 
\textbf{Режим 2}  & Индикаторная буква: Л \\ 
& ОБЕСПЕЧЕНИЕ ГОСУДАРСТВЕННОЙ ПОЛИТИКИ РОССИЙСКОЙ \\ 
\textbf{Режим 4}  & ФЕДЕРАЦИИ СВЯЗАННОЕ С ДОВЕДЕНИЕМ \\ 
\textbf{Режим 5}  & Пароль: МЕЖДУНАРОДНОЙ \\ 
& ДО РОССИЙСКОЙ И МЕЖДУНАРОДНОЙ \\ 
	\end{tabular} 
\end{table}

\end{solution}
\begin{exercise}\begin{table}[H]
	\centering
	\begin{tabular}{r l}\textbf{Режим 1}  & ОЩЯЪЗЗЫРМЕЕЙЙБ СВГГЫХФБЯЯФ ХНТЭНОГСЦЙ Ю \\ 
\textbf{Режим 2}  & ЕОКФГЫЮЮЮУЫДЪЪЫ ГВБЯЩЬСК УЯТАООЬЗКЖ ИЕЪЪШБКГЩ \\ 
\textbf{Режим 3}  & Индикаторная буква: П \\ 
& ЗШ ИТШТЖГЮБЮЮМ ЖПЩЩЗРЛ ДГ \\ 
\textbf{Режим 4}  & ЭдщпуэКбшлряЭвэчурЬкгйдсСфпрьъАпйччщТъждцъ \\ 
\textbf{Режим 5}  & Пароль: МЕЖДУНАРОДНОЙ \\ 
& З КХЧРЙМРРФЯБИО ЖЭЖЦБ Ч \\ 
	\end{tabular} 
\end{table}

\end{exercise}
\begin{solution}
\begin{table}[H]
	\centering
	\begin{tabular}{r l}\textbf{Режим 1}  & ОБЩЕСТВЕННОСТИ ДОСТОВЕРНОЙ ИНФОРМАЦИИ О \\ 
\textbf{Режим 2}  & ГОСУДАРСТВЕННОЙ ПОЛИТИКЕ РОССИЙСКОЙ ФЕДЕРАЦИИ \\ 
\textbf{Режим 2}  & Индикаторная буква: П \\ 
& ЕЕ ОФИЦИАЛЬНОЙ ПОЗИЦИИ ПО \\ 
\textbf{Режим 4}  & СОЦИАЛЬНО ЗНАЧИМЫМ СОБЫТИЯМ РОССИЙСКОЙ \\ 
\textbf{Режим 5}  & Пароль: МЕЖДУНАРОДНОЙ \\ 
& И МЕЖДУНАРОДНОЙ ЖИЗНИ С \\ 
	\end{tabular} 
\end{table}

\end{solution}
\begin{exercise}\begin{table}[H]
	\centering
	\begin{tabular}{r l}\textbf{Режим 1}  & ОЩДОЮЩФРМВБГ АГДДДРО КЫЮКЗПИ К \\ 
\textbf{Режим 2}  & ЫВСБМБДН ЧНЬСЙГКККВГПЦИЕ ВДЧЪДРФЙЯЫУТЙС ТЗАТЬЬДО. \\ 
\textbf{Режим 3}  & Индикаторная буква: О \\ 
& ЭАЮ ХАУУХШЖФЮГ РЭЗГЮ ОЮВУЮЫПЙН \\ 
\textbf{Режим 4}  & ХътвируАяжибйчСчусйцпЪецххлцДинпциьГдгзгээВзгзп \\ 
\textbf{Режим 5}  & Пароль: ВОЗМОЖНОСТИ \\ 
& ЦБГНРСНЕРГ ЕВВМЗРЧМЖ РО СЗЬТНИЯГВПГ \\ 
	\end{tabular} 
\end{table}

\end{exercise}
\begin{solution}
\begin{table}[H]
	\centering
	\begin{tabular}{r l}\textbf{Режим 1}  & ОБЕСПЕЧЕНИЕМ ДОСТУПА ГРАЖДАН К \\ 
\textbf{Режим 2}  & ОТКРЫТЫМ ГОСУДАРСТВЕННЫМ ИНФОРМАЦИОННЫМ РЕСУРСАМ. \\ 
\textbf{Режим 2}  & Индикаторная буква: О \\ 
& ДЛЯ ДОСТИЖЕНИЯ ЭТОГО ТРЕБУЕТСЯ \\ 
\textbf{Режим 4}  & УКРЕПЛЯТЬ ГОСУДАРСТВЕННЫЕ СРЕДСТВА МАССОВОЙ \\ 
\textbf{Режим 5}  & Пароль: ВОЗМОЖНОСТИ \\ 
& ИНФОРМАЦИИ РАСШИРЯТЬ ИХ ВОЗМОЖНОСТИ \\ 
	\end{tabular} 
\end{table}

\end{solution}
\begin{exercise}\begin{table}[H]
	\centering
	\begin{tabular}{r l}\textbf{Режим 1}  & ПЙ ПАДТЬЪСЫИБУУЯЛ ЗСЧЮЧЧКЖВ ЕЗХХЖЛВЧММВ \\ 
\textbf{Режим 2}  & ЫЩЙВЩБОДИН ОЯ АЬЯСШШНЮПИ Й \\ 
\textbf{Режим 3}  & Индикаторная буква: Б \\ 
& ЧШДПЭПКЧ СЩЪДНОЛЛЛШОЬИГУ ШКСХКЮРНЕЧЦЙНШ ЙУМЙЪЪРФ \\ 
\textbf{Режим 4}  & РзшщтВкцфбЬэччуРпкйщЕфэьсМгзууЕэюдзЖбышхХыылэЙи \\ 
\textbf{Режим 5}  & Пароль: НАЦИОНАЛЬНЫХ \\ 
& БЧСЪТХВЮЯНЧЯЭ. ЖЯТЩХЯ ОДЕОХЬДЙЭЦНЯ ЖЬЧЕЛЩАЭКБОЗ \\ 
	\end{tabular} 
\end{table}

\end{exercise}
\begin{solution}
\begin{table}[H]
	\centering
	\begin{tabular}{r l}\textbf{Режим 1}  & ПО СВОЕВРЕМЕННОМУ ДОВЕДЕНИЮ ДОСТОВЕРНОЙ \\ 
\textbf{Режим 2}  & ИНФОРМАЦИИ ДО РОССИЙСКИХ И \\ 
\textbf{Режим 2}  & Индикаторная буква: Б \\ 
& ОТКРЫТЫХ ГОСУДАРСТВЕННЫХ ИНФОРМАЦИОННЫХ РЕСУРСОВ \\ 
\textbf{Режим 4}  & ПОВЫСИТЬ ЭФФЕКТИВНОСТЬ ИХ ХОЗЯЙСТВЕННОГО \\ 
\textbf{Режим 5}  & Пароль: НАЦИОНАЛЬНЫХ \\ 
& ИСПОЛЬЗОВАНИЯ. ТРЕТЬЯ СОСТАВЛЯЮЩАЯ НАЦИОНАЛЬНЫХ \\ 
	\end{tabular} 
\end{table}

\end{solution}
\begin{exercise}\begin{table}[H]
	\centering
	\begin{tabular}{r l}\textbf{Режим 1}  & ИАЫОАШЯНЙ НКЭФЦЦОЙЧД ЗЛББМИЧУТ Э \\ 
\textbf{Режим 2}  & ЫЩЙВЩБОДИТЯССК ЯТЧНЦ ДЖЖЫИЫЩШ Б \\ 
\textbf{Режим 3}  & Индикаторная буква: Ш \\ 
& ЛКЮП АЦЗЦШЬЖЙ ФЗЩЗГЦВЙМЭЮ ЛГЦЩГЫХЕЬУАЯЕЛ \\ 
\textbf{Режим 4}  & КъьмлрбКрнатръУвфвюшыКьллртаФрьеъдгРйййюэшУжйукъъ \\ 
\textbf{Режим 5}  & Пароль: ИНДУСТРИИ \\ 
& С МФЙ ЗГХВТ ЫЩЦЗЦЮОЫЫ \\ 
	\end{tabular} 
\end{table}

\end{exercise}
\begin{solution}
\begin{table}[H]
	\centering
	\begin{tabular}{r l}\textbf{Режим 1}  & ИНТЕРЕСОВ РОССИЙСКОЙ ФЕДЕРАЦИИ В \\ 
\textbf{Режим 2}  & ИНФОРМАЦИОННОЙ СФЕРЕ ВКЛЮЧАЕТ В \\ 
\textbf{Режим 2}  & Индикаторная буква: Ш \\ 
& СЕБЯ РАЗВИТИЕ СОВРЕМЕННЫХ ИНФОРМАЦИОННЫХ \\ 
\textbf{Режим 4}  & ТЕХНОЛОГИЙ ОТЕЧЕСТВЕННОЙ ИНДУСТРИИ ИНФОРМАЦИИ \\ 
\textbf{Режим 5}  & Пароль: ИНДУСТРИИ \\ 
& В ТОМ ЧИСЛЕ ИНДУСТРИИ \\ 
	\end{tabular} 
\end{table}

\end{solution}
\begin{exercise}\begin{table}[H]
	\centering
	\begin{tabular}{r l}\textbf{Режим 1}  & СИПЙЭЭГ ЙЖКЧЖМАЧРЫИЧУТ ЪИТЭСТСЬБИОХЕТЖЧ П \\ 
\textbf{Режим 2}  & ЦЯНУУ ВФИВЫФЪХСКШ ЬНСНЦЪЗЗХХГЕ ЫЙЮЧЙУГЩЯИР \\ 
\textbf{Режим 3}  & Индикаторная буква: К \\ 
& ЩЖБЯШ КХ ААЬЖАОТЖЙЧ Ц \\ 
\textbf{Режим 4}  & ШоцгукчЧбяэатяНяйлчббЖяж \\ 
\textbf{Режим 5}  & Пароль: МИРОВОЙ \\ 
& ФЫОЫГЫХ ЬХЮЫЧ О АМРЕН \\ 
	\end{tabular} 
\end{table}

\end{exercise}
\begin{solution}
\begin{table}[H]
	\centering
	\begin{tabular}{r l}\textbf{Режим 1}  & СРЕДСТВ ИНФОРМАТИЗАЦИИ ТЕЛЕКОММУНИКАЦИИ И \\ 
\textbf{Режим 2}  & СВЯЗИ ОБЕСПЕЧЕНИЕ ПОТРЕБНОСТЕЙ ВНУТРЕННЕГО \\ 
\textbf{Режим 2}  & Индикаторная буква: К \\ 
& РЫНКА ЕЕ ПРОДУКЦИЕЙ И \\ 
\textbf{Режим 4}  & ВЫХОД ЭТОЙ ПРОДУКЦИИ НА \\ 
\textbf{Режим 5}  & Пароль: МИРОВОЙ \\ 
& МИРОВОЙ РЫНОК А ТАКЖЕ \\ 
	\end{tabular} 
\end{table}

\end{solution}
\begin{exercise}\begin{table}[H]
	\centering
	\begin{tabular}{r l}\textbf{Режим 1}  & ОЩДОЮЩФРМВБ ЪИЛЩЩБИБЬЮ ЫАРАЦНЭЭИИЧ П \\ 
\textbf{Режим 2}  & ОЪЗОЖЖЙТ. А ЙЪЬЪСЫИБУЦЩ ЛУНЬПЗДА \\ 
\textbf{Режим 3}  & Индикаторная буква: А \\ 
& ЖМГФРГ ВФ ЗГЫС УЫЯЯЖШ \\ 
\textbf{Режим 4}  & ЛчюджюЮбкдюлЕщниэмЩюмшбфХлснхыЮнж \\ 
\textbf{Режим 5}  & Пароль: МИРОВОЙ \\ 
& ЭИФАЦ С АЖЧЮСЧС. СНМХСР \\ 
	\end{tabular} 
\end{table}

\end{exercise}
\begin{solution}
\begin{table}[H]
	\centering
	\begin{tabular}{r l}\textbf{Режим 1}  & ОБЕСПЕЧЕНИЕ НАКОПЛЕНИЯ СОХРАННОСТИ И \\ 
\textbf{Режим 2}  & РЕСУРСОВ. В СОВРЕМЕННЫХ УСЛОВИЯХ \\ 
\textbf{Режим 2}  & Индикаторная буква: А \\ 
& ТОЛЬКО НА ЭТОЙ ОСНОВЕ \\ 
\textbf{Режим 4}  & МОЖНО РЕШАТЬ ПРОБЛЕМЫ СОЗДАНИЯ \\ 
\textbf{Режим 5}  & Пароль: МИРОВОЙ \\ 
& НАУКИ И ТЕХНИКИ. РОССИЯ \\ 
	\end{tabular} 
\end{table}

\end{solution}
\begin{exercise}\begin{table}[H]
	\centering
	\begin{tabular}{r l}\textbf{Режим 1}  & ДШПЪПУ ЩАЕЬЦК ЯЩЪЪЛЯУУК СОТТИ \\ 
\textbf{Режим 2}  & ЦМТЯУ ЛАВЫХЦЩ ЬФШШСНЙ ХЖЮКЗБПГМШЧЙЕДДЛ \\ 
\textbf{Режим 3}  & Индикаторная буква: Н \\ 
& Б ЫВЫРЖЮРКУССК ЬЬЭХЛТЮЕЮЖЖШШГ. ЫЪЬ \\ 
\textbf{Режим 4}  & УрпюшХяьиыКахзъАйбйжЮбкшрЪцингБюбълУъыушЧм \\ 
\textbf{Режим 5}  & Пароль: ЕДИНОГО \\ 
& Ш ИЪЬНФЙЧКММКХКЕЬК ББЬРЕИЙДГЮЩЯНЦ АЭЧБЭЙЪ \\ 
	\end{tabular} 
\end{table}

\end{exercise}
\begin{solution}
\begin{table}[H]
	\centering
	\begin{tabular}{r l}\textbf{Режим 1}  & ДОЛЖНА ЗАНЯТЬ ДОСТОЙНОЕ МЕСТО \\ 
\textbf{Режим 2}  & СРЕДИ МИРОВЫХ ЛИДЕРОВ МИКРОЭЛЕКТРОННОЙ \\ 
\textbf{Режим 2}  & Индикаторная буква: Н \\ 
& И КОМПЬЮТЕРНОЙ ПРОМЫШЛЕННОСТИ. ДЛЯ \\ 
\textbf{Режим 4}  & ДОСТИЖЕНИЯ ЭТОГО ТРЕБУЕТСЯ РАЗВИВАТЬ \\ 
\textbf{Режим 5}  & Пароль: ЕДИНОГО \\ 
& И СОВЕРШЕНСТВОВАТЬ ИНФРАСТРУКТУРУ ЕДИНОГО \\ 
	\end{tabular} 
\end{table}

\end{solution}
\begin{exercise}\begin{table}[H]
	\centering
	\begin{tabular}{r l}\textbf{Режим 1}  & ИАРЯАНЙБШФХООЪШ ШШЧЮЮПЯДЙЙЬЭ ГРЩВЬЬЫИЯФ БЮЧЧНЪЯПЦ \\ 
\textbf{Режим 2}  & ОЛДЛЪБАЧЭ ДЙАЧСГГКЭТАЛЦ КИЖАЬЬЭЧР ЦЮФЖЮЙТОВЕЪГОЦ \\ 
\textbf{Режим 3}  & Индикаторная буква: Щ \\ 
& НКЧКВ Д ЧЦУЭОЬЙХ ХМЕИАВИЗЯЯТТЪ \\ 
\textbf{Режим 4}  & НднуьбьЮюлцубтУъгыпрюАжуйупйЫзземзйИъл \\ 
\textbf{Режим 5}  & Пароль: СРЕДСТВ \\ 
& ЖДППЙТЬРЙ ОФДДЭЙДФЬЕЖОЕХЭЕДИИ ЦОЧЦЦЮГ Р \\ 
	\end{tabular} 
\end{table}

\end{exercise}
\begin{solution}
\begin{table}[H]
	\centering
	\begin{tabular}{r l}\textbf{Режим 1}  & ИНФОРМАЦИОННОГО ПРОСТРАНСТВА РОССИЙСКОЙ ФЕДЕРАЦИИ \\ 
\textbf{Режим 2}  & РАЗВИВАТЬ ОТЕЧЕСТВЕННУЮ ИНДУСТРИЮ ИНФОРМАЦИОННЫХ \\ 
\textbf{Режим 2}  & Индикаторная буква: Щ \\ 
& УСЛУГ И ПОВЫШАТЬ ЭФФЕКТИВНОСТЬ \\ 
\textbf{Режим 4}  & РАЗВИВАТЬ ПРОИЗВОДСТВО В РОССИЙСКОЙ \\ 
\textbf{Режим 5}  & Пароль: СРЕДСТВ \\ 
& ФЕДЕРАЦИИ КОНКУРЕНТОСПОСОБНЫХ СРЕДСТВ И \\ 
	\end{tabular} 
\end{table}

\end{solution}
\begin{exercise}\begin{table}[H]
	\centering
	\begin{tabular}{r l}\textbf{Режим 1}  & СШИИЙШ ЦЮФЖЮЙТЖВЛЬЖБУ ДЬУНЯУЯСЫЬХКМУЮЖ Ч \\ 
\textbf{Режим 2}  & ЦЯНУУ ГКЩЙЬЫЖЫМ БРЙССЮЦ ФОКХМЕ \\ 
\textbf{Режим 3}  & Индикаторная буква: П \\ 
& Ж НЮЮПСХГКЗЪЖЖЕ ЕЦЙЙБМИЧУТ ВВЫЬНЗАЗКТЮКЧШ \\ 
\textbf{Режим 4}  & ФодтйъгсЦюатчхахЯпчыд \\ 
\textbf{Режим 5}  & Пароль: ИССЛЕДОВАНИЙ \\ 
& ПЧДГЕХНШЖНВЪГФЮ М ННСЧДНАНХЮ РОЬОНДЙЬИНЫВ \\ 
	\end{tabular} 
\end{table}

\end{exercise}
\begin{solution}
\begin{table}[H]
	\centering
	\begin{tabular}{r l}\textbf{Режим 1}  & СИСТЕМ ИНФОРМАТИЗАЦИИ ТЕЛЕКОММУНИКАЦИИ И \\ 
\textbf{Режим 2}  & СВЯЗИ РАСШИРЯТЬ УЧАСТИЕ РОССИИ \\ 
\textbf{Режим 2}  & Индикаторная буква: П \\ 
& В МЕЖДУНАРОДНОЙ КООПЕРАЦИИ ПРОИЗВОДИТЕЛЕЙ \\ 
\textbf{Режим 4}  & ЭТИХ СРЕДСТВ И СИСТЕМ \\ 
\textbf{Режим 5}  & Пароль: ИССЛЕДОВАНИЙ \\ 
& ФУНДАМЕНТАЛЬНЫХ И ПРИКЛАДНЫХ ИССЛЕДОВАНИЙ \\ 
	\end{tabular} 
\end{table}

\end{solution}
\begin{exercise}\begin{table}[H]
	\centering
	\begin{tabular}{r l}\textbf{Режим 1}  & РЗНБЖЖСТИХ Й ИСЙКВН МЧХПЧЕЯПЛБНПТА \\ 
\textbf{Режим 2}  & ЮЛЪБВЪВРЕЛЯАЗЪНЭ Ф ТЧМШШ. УЦНЪЕОХЫУ \\ 
\textbf{Режим 3}  & Индикаторная буква: П \\ 
& ТЬЯЯЕМЗЩРАРЫ ЧБКГЙЕСЩКЩЗЙ АЫЩЭЫКБЯЖ ЯИСЬЮЮЭЖЗЦ \\ 
\textbf{Режим 4}  & ШщфкфЖужщэУъыъцНцгдвУукъпВшшзпКгеьщАл \\ 
\textbf{Режим 5}  & Пароль: ВКЛЮЧАЕТ \\ 
& ГУПЧЩАЧЮ Г ГЪШП ВЕЭЦЪК \\ 
	\end{tabular} 
\end{table}

\end{exercise}
\begin{solution}
\begin{table}[H]
	\centering
	\begin{tabular}{r l}\textbf{Режим 1}  & РАЗРАБОТОК В СФЕРАХ ИНФОРМАТИЗАЦИИ \\ 
\textbf{Режим 2}  & ТЕЛЕКОММУНИКАЦИИ И СВЯЗИ. ЧЕТВЕРТАЯ \\ 
\textbf{Режим 2}  & Индикаторная буква: П \\ 
& СОСТАВЛЯЮЩАЯ НАЦИОНАЛЬНЫХ ИНТЕРЕСОВ РОССИЙСКОЙ \\ 
\textbf{Режим 4}  & ФЕДЕРАЦИИ В ИНФОРМАЦИОННОЙ СФЕРЕ \\ 
\textbf{Режим 5}  & Пароль: ВКЛЮЧАЕТ \\ 
& ВКЛЮЧАЕТ В СЕБЯ ЗАЩИТУ \\ 
	\end{tabular} 
\end{table}

\end{solution}
\begin{exercise}\begin{table}[H]
	\centering
	\begin{tabular}{r l}\textbf{Режим 1}  & ИАРЯАНЙБШФХОБИ ОЪЗОЖЖЙТ ЕЦ ГАДФЛТЕЬУАЭТСЧЙЭФФМХ \\ 
\textbf{Режим 2}  & ЦКЭЭЭЗЛ ЖЫВЖЦЫЗУГБЯ ЭЬСЫЫШЫЯЯТТО ЗФЯКФШВИМЖПЦИЗ \\ 
\textbf{Режим 3}  & Индикаторная буква: П \\ 
& К ТЮКЧЗКЗШЭЮМЙБКГЙЕДКЧ ЕУЪЪИУ ЯШЬ \\ 
\textbf{Режим 4}  & ЮрфкбюКяэьщлИцйхчйЖжиэ \\ 
\textbf{Режим 5}  & Пароль: ТЕРРИТОРИИ \\ 
& ЖЭМЪИБОДЩХО ФР КТЧРЙЙЪЧШЙ ОЪЖВЙЫ. \\ 
	\end{tabular} 
\end{table}

\end{exercise}
\begin{solution}
\begin{table}[H]
	\centering
	\begin{tabular}{r l}\textbf{Режим 1}  & ИНФОРМАЦИОННЫХ РЕСУРСОВ ОТ НЕСАНКЦИОНИРОВАННОГО \\ 
\textbf{Режим 2}  & ДОСТУПА ОБЕСПЕЧЕНИЕ БЕЗОПАСНОСТИ ИНФОРМАЦИОННЫХ \\ 
\textbf{Режим 2}  & Индикаторная буква: П \\ 
& И ТЕЛЕКОММУНИКАЦИОННЫХ СИСТЕМ КАК \\ 
\textbf{Режим 4}  & УЖЕ РАЗВЕРНУТЫХ ТАК И \\ 
\textbf{Режим 5}  & Пароль: ТЕРРИТОРИИ \\ 
& СОЗДАВАЕМЫХ НА ТЕРРИТОРИИ РОССИИ. \\ 
	\end{tabular} 
\end{table}

\end{solution}
\begin{exercise}\begin{table}[H]
	\centering
	\begin{tabular}{r l}\textbf{Режим 1}  & В ТЗЪО ОЫГИЧ ЩЯВФЕЫФИЯА \\ 
\textbf{Режим 2}  & ЯЬПЩЬЮИЛ ГЕМЮЮБЮЙЙППФ ЫЩЙВЩБОДИТЯСДЫ ЯЗЬЬЦЗ \\ 
\textbf{Режим 3}  & Индикаторная буква: Щ \\ 
& ЪЧЧБНБТ ЖЛЧР ЙСЭТТ ВВННКК \\ 
\textbf{Режим 4}  & РщхдгшЪдцчкфЯяпемпЬцфеумПжквьдРнхдкдЗс \\ 
\textbf{Режим 5}  & Пароль: ИНФОРМАЦИОННЫХ \\ 
& МЬКЬЙ З ЩИПЫДВЫМЫГРБТУ ЧХМЩАВ \\ 
	\end{tabular} 
\end{table}

\end{exercise}
\begin{solution}
\begin{table}[H]
	\centering
	\begin{tabular}{r l}\textbf{Режим 1}  & В ЭТИХ ЦЕЛЯХ НЕОБХОДИМО \\ 
\textbf{Режим 2}  & ПОВЫСИТЬ БЕЗОПАСНОСТЬ ИНФОРМАЦИОННЫХ СИСТЕМ \\ 
\textbf{Режим 2}  & Индикаторная буква: Щ \\ 
& ВКЛЮЧАЯ СЕТИ СВЯЗИ ПРЕЖДЕ \\ 
\textbf{Режим 4}  & ВСЕГО БЕЗОПАСНОСТЬ ПЕРВИЧНЫХ СЕТЕЙ \\ 
\textbf{Режим 5}  & Пароль: ИНФОРМАЦИОННЫХ \\ 
& СВЯЗИ И ИНФОРМАЦИОННЫХ СИСТЕМ \\ 
	\end{tabular} 
\end{table}

\end{solution}
\begin{exercise}\begin{table}[H]
	\centering
	\begin{tabular}{r l}\textbf{Режим 1}  & ФГВВЧАЪЭЪХП ЩГНХББО ОАУБЮЙЬЬЬЕЙЗШШЙ РЙУЖЖВ \\ 
\textbf{Режим 2}  & ЫЛХЮЯЯЖ ЖИСЯЦЪФФФЩЪЧППЪ УЪСЙЙБ ДЕФЧНЯРТШ \\ 
\textbf{Режим 3}  & Индикаторная буква: Ы \\ 
& ШЧЮЖЩЩЦВЪЫ ЙСИИРЗЪНЭ ЛКИЙЭЬФЕХ-ЖХДГЕЗЦЦГ В \\ 
\textbf{Режим 4}  & ЦпцифГхдэюХлащдТрчжзТрчиоШунюьВкфгеЛжжюй \\ 
\textbf{Режим 5}  & Пароль: ДЕЯТЕЛЬНОСТИ \\ 
& ЦЧСЮЧПФЩЫЦЮЫ Д СЯДЖЪ ЙБВПРЩ \\ 
	\end{tabular} 
\end{table}

\end{exercise}
\begin{solution}
\begin{table}[H]
	\centering
	\begin{tabular}{r l}\textbf{Режим 1}  & ФЕДЕРАЛЬНЫХ ОРГАНОВ ГОСУДАРСТВЕННОЙ ВЛАСТИ \\ 
\textbf{Режим 2}  & ОРГАНОВ ГОСУДАРСТВЕННОЙ ВЛАСТИ СУБЪЕКТОВ \\ 
\textbf{Режим 2}  & Индикаторная буква: Ы \\ 
& РОССИЙСКОЙ ФЕДЕРАЦИИ ФИНАНСОВО-КРЕДИТНОЙ И \\ 
\textbf{Режим 4}  & БАНКОВСКОЙ СФЕР СФЕРЫ ХОЗЯЙСТВЕННОЙ \\ 
\textbf{Режим 5}  & Пароль: ДЕЯТЕЛЬНОСТИ \\ 
& ДЕЯТЕЛЬНОСТИ А ТАКЖЕ СИСТЕМ \\ 
	\end{tabular} 
\end{table}

\end{solution}
\begin{exercise}\begin{table}[H]
	\centering
	\begin{tabular}{r l}\textbf{Режим 1}  & И ЫУОШААЦ ШКСХКЮРХЕЪУХЩВ АДЪДМЯЬЫСШ \\ 
\textbf{Режим 2}  & Ы ЬЩСЛЫЫС РКВСКФО СШИИЙШ \\ 
\textbf{Режим 3}  & Индикаторная буква: Ч \\ 
& ЫЯЯЙЦХПХЧВ ГШЙЗМУМЩ В ЕМЦТУСЫ \\ 
\textbf{Режим 4}  & СнщегетИыютмрыЯмйяпеэУйъъяепЯемдвоыРхпй \\ 
\textbf{Режим 5}  & Пароль: РАЗВИТИЕ \\ 
& ЩАНШХМЫ НОЙЬЙСЦАЬКУЗЮЫ ГГСДМРЦПГЖШОЙХВЙЬ ЕЕМУЬФЫР \\ 
	\end{tabular} 
\end{table}

\end{exercise}
\begin{solution}
\begin{table}[H]
	\centering
	\begin{tabular}{r l}\textbf{Режим 1}  & И СРЕДСТВ ИНФОРМАТИЗАЦИИ ВООРУЖЕНИЯ \\ 
\textbf{Режим 2}  & И ВОЕННОЙ ТЕХНИКИ СИСТЕМ \\ 
\textbf{Режим 2}  & Индикаторная буква: Ч \\ 
& УПРАВЛЕНИЯ ВОЙСКАМИ И ОРУЖИЕМ \\ 
\textbf{Режим 4}  & ЭКОЛОГИЧЕСКИ ОПАСНЫМИ И ЭКОНОМИЧЕСКИ \\ 
\textbf{Режим 5}  & Пароль: РАЗВИТИЕ \\ 
& ВАЖНЫМИ ПРОИЗВОДСТВАМИ ИНТЕНСИФИЦИРОВАТЬ РАЗВИТИЕ \\ 
	\end{tabular} 
\end{table}

\end{solution}
\begin{exercise}\begin{table}[H]
	\centering
	\begin{tabular}{r l}\textbf{Режим 1}  & ОФДИГШШБЛЕДДЯГ ГГРЯЬХУХББЖЦ ЙИНДЭГЭШТЭ Е \\ 
\textbf{Режим 2}  & ЯЯИПИДОЗЗУН ОШЩРЖЖТ ЛЬЗБЕШ АЫЪБЫЯЖВКХ \\ 
\textbf{Режим 3}  & Индикаторная буква: Э \\ 
& У ЛЬГЫФУЗ ИЯЬТСНХЪ ЧГ \\ 
\textbf{Режим 4}  & ЩпьлиЩимжэСрадеЛохтбМнйжсНъажбНмхунЯквчж \\ 
\textbf{Режим 5}  & Пароль: ГОСУДАРСТВЕННУЮ \\ 
& ЭКБРПМЙВ ЖЗЬМНЫСЩШХЪА БШЦШДЕДЕШЗЙБДЗГ ЖРВЙЭ \\ 
	\end{tabular} 
\end{table}

\end{exercise}
\begin{solution}
\begin{table}[H]
	\centering
	\begin{tabular}{r l}\textbf{Режим 1}  & ОТЕЧЕСТВЕННОГО ПРОИЗВОДСТВА АППАРАТНЫХ И \\ 
\textbf{Режим 2}  & ПРОГРАММНЫХ СРЕДСТВ ЗАЩИТЫ ИНФОРМАЦИИ \\ 
\textbf{Режим 2}  & Индикаторная буква: Э \\ 
& И МЕТОДОВ КОНТРОЛЯ ЗА \\ 
\textbf{Режим 4}  & ИХ ЭФФЕКТИВНОСТЬЮ ОБЕСПЕЧИТЬ ЗАЩИТУ \\ 
\textbf{Режим 5}  & Пароль: ГОСУДАРСТВЕННУЮ \\ 
& СВЕДЕНИЙ СОСТАВЛЯЮЩИХ ГОСУДАРСТВЕННУЮ ТАЙНУ \\ 
	\end{tabular} 
\end{table}

\end{solution}
\begin{exercise}\begin{table}[H]
	\centering
	\begin{tabular}{r l}\textbf{Режим 1}  & РЗЛДЭТЦТЪ ФЧЧЙИЗВОЮЩППР ЧМПМЦЯВТЦИВВОТ УЯТАООЬЗКЖ \\ 
\textbf{Режим 2}  & МЬННЛШГЭФ Ж ЬЦКЕИИЧ КВМВЕЗЪА \\ 
\textbf{Режим 3}  & Индикаторная буква: Ь \\ 
& Л ЬАТВВХВУУОА ФТСИКЩЖКДЩШЙЫ ЕПОЦПДСЮЫГВРЮЕ \\ 
\textbf{Режим 4}  & ПазтбатЗбхылбвТгбцлръЪниежгьЧштякотЖжфшчфмКэклах \\ 
\textbf{Режим 5}  & Пароль: ИНФОРМАЦИОННОЙ \\ 
& ДДВЩЙЦЙПЪРЕМКЙ С БРЪШЬЕЦЖСГЩБИЛ ЕМНОЖ. \\ 
	\end{tabular} 
\end{table}

\end{exercise}
\begin{solution}
\begin{table}[H]
	\centering
	\begin{tabular}{r l}\textbf{Режим 1}  & РАСШИРЯТЬ МЕЖДУНАРОДНОЕ СОТРУДНИЧЕСТВО РОССИЙСКОЙ \\ 
\textbf{Режим 2}  & ФЕДЕРАЦИИ В ОБЛАСТИ РАЗВИТИЯ \\ 
\textbf{Режим 2}  & Индикаторная буква: Ь \\ 
& И БЕЗОПАСНОГО ИСПОЛЬЗОВАНИЯ ИНФОРМАЦИОННЫХ \\ 
\textbf{Режим 4}  & РЕСУРСОВ ПРОТИВОДЕЙСТВИЯ УГРОЗЕ РАЗВЯЗЫВАНИЯ \\ 
\textbf{Режим 5}  & Пароль: ИНФОРМАЦИОННОЙ \\ 
& ПРОТИВОБОРСТВА В ИНФОРМАЦИОННОЙ СФЕРЕ. \\ 
	\end{tabular} 
\end{table}

\end{solution}
\begin{exercise}\begin{table}[H]
	\centering
	\begin{tabular}{r l}\textbf{Режим 1}  & ПЙ ПАДТЫ ЩЭЧИЯ УШУУЧЖЭЮЭФФННП \\ 
\textbf{Режим 2}  & ЗЫПМЛК ЫЩЙВЩБОДИТЯССК ЦЖЮФФГФШШООД ЖЦЧПРРМЛЩУ \\ 
\textbf{Режим 3}  & Индикаторная буква: Ы \\ 
& КЩРРПЯШЛЫ ЪЩЬЩХЬФФСЧЦУАМ ЭЕ ИЗЙЕЭЫСЕЩ \\ 
\textbf{Режим 4}  & ЛвмгакНсдгуоЩжозфэЛмхкнмЯмееэдЙьсйпйТм \\ 
\textbf{Режим 5}  & Пароль: ЧЕЛОВЕКА \\ 
& Ф ЬВЫЩЭИАЬ БЪТЗЙЬЪЧ Ш \\ 
	\end{tabular} 
\end{table}

\end{exercise}
\begin{solution}
\begin{table}[H]
	\centering
	\begin{tabular}{r l}\textbf{Режим 1}  & ПО СВОЕЙ ОБЩЕЙ НАПРАВЛЕННОСТИ \\ 
\textbf{Режим 2}  & УГРОЗЫ ИНФОРМАЦИОННОЙ БЕЗОПАСНОСТИ РОССИЙСКОЙ \\ 
\textbf{Режим 2}  & Индикаторная буква: Ы \\ 
& ФЕДЕРАЦИИ ПОДРАЗДЕЛЯЮТСЯ НА СЛЕДУЮЩИЕ \\ 
\textbf{Режим 4}  & ВИДЫ УГРОЗЫ КОНСТИТУЦИОННЫМ ПРАВАМ \\ 
\textbf{Режим 5}  & Пароль: ЧЕЛОВЕКА \\ 
& И СВОБОДАМ ЧЕЛОВЕКА И \\ 
	\end{tabular} 
\end{table}

\end{solution}
\begin{exercise}\begin{table}[H]
	\centering
	\begin{tabular}{r l}\textbf{Режим 1}  & ГЗЫГРТМВДЬ С ГНЫКЩЩЬ КРВЯЖИИО \\ 
\textbf{Режим 2}  & КФХИО З ШКСХКЮРНЕЧЦЙЙЩ УУНМЯЫШЫЫРРН \\ 
\textbf{Режим 3}  & Индикаторная буква: В \\ 
& ЦЮГЕЛЪЛЮЯГФГГРЕ НВГЫБТШЯИУ О НМЫЦЭЭГДЖЧЧЙШ \\ 
\textbf{Режим 4}  & ПтсхьпьЖэполваЭивщрмиГхпфхцмЙюноесиЯппвв \\ 
\textbf{Режим 5}  & Пароль: ФЕДЕРАЦИИ \\ 
& ЩЭОШКЩФЫ ЕВЬИШЦЧФЪЛ МЧЦЧОАЛЫЫ ЕЦНЭМН \\ 
	\end{tabular} 
\end{table}

\end{exercise}
\begin{solution}
\begin{table}[H]
	\centering
	\begin{tabular}{r l}\textbf{Режим 1}  & ГРАЖДАНИНА В ОБЛАСТИ ДУХОВНОЙ \\ 
\textbf{Режим 2}  & ЖИЗНИ И ИНФОРМАЦИОННОЙ ДЕЯТЕЛЬНОСТИ \\ 
\textbf{Режим 2}  & Индикаторная буква: В \\ 
& ИНДИВИДУАЛЬНОМУ ГРУППОВОМУ И ОБЩЕСТВЕННОМУ \\ 
\textbf{Режим 4}  & СОЗНАНИЮ ДУХОВНОМУ ВОЗРОЖДЕНИЮ РОССИИ \\ 
\textbf{Режим 5}  & Пароль: ФЕДЕРАЦИИ \\ 
& ПОЛИТИКИ РОССИЙСКОЙ ФЕДЕРАЦИИ УГРОЗЫ \\ 
	\end{tabular} 
\end{table}

\end{solution}
\begin{exercise}\begin{table}[H]
	\centering
	\begin{tabular}{r l}\textbf{Режим 1}  & РЗНЗЭБКМ ИЯЧУЦЭЭГДЖЧЧД ГМБЧЙЙДУТ АЫЪБЫЯЖВКХ \\ 
\textbf{Режим 2}  & ГППУЭУА ЩЕУПММЪТО ЩРЭКЫЫЮ КИУНИХЪНПЙРНЕД \\ 
\textbf{Режим 3}  & Индикаторная буква: Д \\ 
& ЬЦШМЧШЧПОЦГЪЯШЛЫ Б ДНХСС БЗКБАЗРЮЭШГ \\ 
\textbf{Режим 4}  & УчпшжвбцУпюшвдыцАюжплммлХэлвжытхЗх \\ 
\textbf{Режим 5}  & Пароль: ПРОДУКЦИИ \\ 
& ЗД ЛНПИУРЖКЙ С МЬМЬАЦ \\ 
	\end{tabular} 
\end{table}

\end{exercise}
\begin{solution}
\begin{table}[H]
	\centering
	\begin{tabular}{r l}\textbf{Режим 1}  & РАЗВИТИЮ ОТЕЧЕСТВЕННОЙ ИНДУСТРИИ ИНФОРМАЦИИ \\ 
\textbf{Режим 2}  & ВКЛЮЧАЯ ИНДУСТРИЮ СРЕДСТВ ИНФОРМАТИЗАЦИИ \\ 
\textbf{Режим 2}  & Индикаторная буква: Д \\ 
& ТЕЛЕКОММУНИКАЦИИ И СВЯЗИ ОБЕСПЕЧЕНИЮ \\ 
\textbf{Режим 4}  & ПОТРЕБНОСТЕЙ ВНУТРЕННЕГО РЫНКА В \\ 
\textbf{Режим 5}  & Пароль: ПРОДУКЦИИ \\ 
& ЕЕ ПРОДУКЦИИ И ВЫХОДУ \\ 
	\end{tabular} 
\end{table}

\end{solution}
\begin{exercise}\begin{table}[H]
	\centering
	\begin{tabular}{r l}\textbf{Режим 1}  & ЭЙГБ ЩЩЛЭЩСГЭФ ЬЦ ЭЗИФЕХЧ \\ 
\textbf{Режим 2}  & ОАЧЧЕ Т ЖЯЩБТ ГНСГРНЕФАЭЦ \\ 
\textbf{Режим 3}  & Индикаторная буква: З \\ 
& УШЬААНШНЗД ИКСКВЮЖЖШШГ Щ ИЖЧТЛЕТЭЫЫКУ \\ 
\textbf{Режим 4}  & ЮробтэйИьтбъдиЭдвщдеуЯвеюмцдЭэпущввСфюр \\ 
\textbf{Режим 5}  & Пароль: СРЕДСТВ \\ 
& ПДХПЩЦЮЪФФЮЩТЬРШФКФО КЙДПИПБ Ц ЦЙЬЬБМ \\ 
	\end{tabular} 
\end{table}

\end{exercise}
\begin{solution}
\begin{table}[H]
	\centering
	\begin{tabular}{r l}\textbf{Режим 1}  & ЭТОЙ ПРОДУКЦИИ НА МИРОВОЙ \\ 
\textbf{Режим 2}  & РЫНОК А ТАКЖЕ ОБЕСПЕЧЕНИЮ \\ 
\textbf{Режим 2}  & Индикаторная буква: З \\ 
& НАКОПЛЕНИЯ СОХРАННОСТИ И ЭФФЕКТИВНОГО \\ 
\textbf{Режим 4}  & УГРОЗЫ БЕЗОПАСНОСТИ ИНФОРМАЦИОННЫХ И \\ 
\textbf{Режим 5}  & Пароль: СРЕДСТВ \\ 
& ТЕЛЕКОММУНИКАЦИОННЫХ СРЕДСТВ И СИСТЕМ \\ 
	\end{tabular} 
\end{table}

\end{solution}
\begin{exercise}\begin{table}[H]
	\centering
	\begin{tabular}{r l}\textbf{Режим 1}  & КМО СПЙ КВМВГЮЦШЖЭЮ ЦИЛ \\ 
\textbf{Режим 2}  & Ы ЕЩАИХКЗФЯМР ЬЦ АЖИНЖНОХЙМ \\ 
\textbf{Режим 3}  & Индикаторная буква: Е \\ 
& ИФНЭПЦ. ФЩШЧГТЕВ ЩДГЕЕТЪЪЕЬУАЯЕИ ССМЙДО \\ 
\textbf{Режим 4}  & БдшлчычЕпекбчхЕэйчоеш \\ 
\textbf{Режим 5}  & Пароль: ДУХОВНОЙ \\ 
& ЙЖХИДНБОКУ Ф ЫЩРОЯЬЪ НГИГКНФД \\ 
	\end{tabular} 
\end{table}

\end{exercise}
\begin{solution}
\begin{table}[H]
	\centering
	\begin{tabular}{r l}\textbf{Режим 1}  & КАК УЖЕ РАЗВЕРНУТЫХ ТАК \\ 
\textbf{Режим 2}  & И СОЗДАВАЕМЫХ НА ТЕРРИТОРИИ \\ 
\textbf{Режим 2}  & Индикаторная буква: Е \\ 
& РОССИИ. УГРОЗАМИ КОНСТИТУЦИОННЫМ ПРАВАМ \\ 
\textbf{Режим 4}  & И СВОБОДАМ ЧЕЛОВЕКА И \\ 
\textbf{Режим 5}  & Пароль: ДУХОВНОЙ \\ 
& ГРАЖДАНИНА В ОБЛАСТИ ДУХОВНОЙ \\ 
	\end{tabular} 
\end{table}

\end{solution}
\begin{exercise}\begin{table}[H]
	\centering
	\begin{tabular}{r l}\textbf{Режим 1}  & ЖЮЧОП Ц ЙЖКЧЖМАЗРДГЩЩУ ИИЗВФСЙССААЗ \\ 
\textbf{Режим 2}  & ЫЩСТЭЯЭЩШСЙССНТ ПИСФКОГШЧХ Д ПУФЫЦЦСАВРРБГ \\ 
\textbf{Режим 3}  & Индикаторная буква: Х \\ 
& ЩЫИТЮЯФЙ ЮТУЭМХХШЭ ЩЮДЮПВЫЫДЛЭ ДЩЪГЯС \\ 
\textbf{Режим 4}  & НвгсйКъхэбЕдсщьХорвяОбзщсЮквкхЛъзлпРуютжХя \\ 
\textbf{Режим 5}  & Пароль: ОРГАНАМИ \\ 
& ЫОЕАЩАФЫ ФШЭЮЯАЬМЩЩЙМЩЙО СТРЭЖБ ЙЬЯОЮГЕБ \\ 
	\end{tabular} 
\end{table}

\end{exercise}
\begin{solution}
\begin{table}[H]
	\centering
	\begin{tabular}{r l}\textbf{Режим 1}  & ЖИЗНИ И ИНФОРМАЦИОННОЙ ДЕЯТЕЛЬНОСТИ \\ 
\textbf{Режим 2}  & ИНДИВИДУАЛЬНОМУ ГРУППОВОМУ И ОБЩЕСТВЕННОМУ \\ 
\textbf{Режим 2}  & Индикаторная буква: Х \\ 
& СОЗНАНИЮ ДУХОВНОМУ ВОЗРОЖДЕНИЮ РОССИИ \\ 
\textbf{Режим 4}  & МОГУТ ЯВЛЯТЬСЯ ПРИНЯТИЕ ФЕДЕРАЛЬНЫМИ \\ 
\textbf{Режим 5}  & Пароль: ОРГАНАМИ \\ 
& ОРГАНАМИ ГОСУДАРСТВЕННОЙ ВЛАСТИ ОРГАНАМИ \\ 
	\end{tabular} 
\end{table}

\end{solution}
\begin{exercise}\begin{table}[H]
	\centering
	\begin{tabular}{r l}\textbf{Режим 1}  & ГЮЗОРТПППЯТЩВВТ ФТЗЛЛЭ УБПЩЖОЬКД КЗХОЕЕЮДЙЩ \\ 
\textbf{Режим 2}  & МЬННЛШГЭФ ЬЬСФЕЬЖДЗУН ШШБЫЦУЭЮ ЬРМЩН \\ 
\textbf{Режим 3}  & Индикаторная буква: К \\ 
& ДПНТЯЖПЪКЫ КНФИИЧЭЭИМЖПЦИЛ ЙЙЬСН У \\ 
\textbf{Режим 4}  & ГэдхъГхмыеЯцялбИигсъФэыфъСпр \\ 
\textbf{Режим 5}  & Пароль: ИНФОРМАЦИОННОЙ \\ 
& АЙЙЫЩЭЙК УСУЩС О ЫЩМЫОФАЛЫЫЩЩЫХ \\ 
	\end{tabular} 
\end{table}

\end{exercise}
\begin{solution}
\begin{table}[H]
	\centering
	\begin{tabular}{r l}\textbf{Режим 1}  & ГОСУДАРСТВЕННОЙ ВЛАСТИ СУБЪЕКТОВ РОССИЙСКОЙ \\ 
\textbf{Режим 2}  & ФЕДЕРАЦИИ НОРМАТИВНЫХ ПРАВОВЫХ АКТОВ \\ 
\textbf{Режим 2}  & Индикаторная буква: К \\ 
& УЩЕМЛЯЮЩИХ КОНСТИТУЦИОННЫЕ ПРАВА И \\ 
\textbf{Режим 4}  & СВОБОДЫ ГРАЖДАН В ОБЛАСТИ \\ 
\textbf{Режим 5}  & Пароль: ИНФОРМАЦИОННОЙ \\ 
& ДУХОВНОЙ ЖИЗНИ И ИНФОРМАЦИОННОЙ \\ 
	\end{tabular} 
\end{table}

\end{solution}
\begin{exercise}\begin{table}[H]
	\centering
	\begin{tabular}{r l}\textbf{Режим 1}  & ДДЫОЩМЬММЦЦЫ ЕЩАИХБЬФ ЯАСССИКШШ ХГ \\ 
\textbf{Режим 2}  & МРВУЬЫТШЧЬХЮ ИЭОЬАМШЙЪ Е ЮУЖЦЦЕЙЙДЭРЬЫСЭ \\ 
\textbf{Режим 3}  & Индикаторная буква: О \\ 
& СУЩТУЬПЛОЗ Й НКЭФЦЦОЙЧД ЗЛББМИЧУТ \\ 
\textbf{Режим 4}  & ЯтчмдоКагбьг \\ 
\textbf{Режим 5}  & Пароль: ИНФОРМАЦИОННОЙ \\ 
& С МВУ ЬЗЧЭТ ВГ \\ 
	\end{tabular} 
\end{table}

\end{exercise}
\begin{solution}
\begin{table}[H]
	\centering
	\begin{tabular}{r l}\textbf{Режим 1}  & ДЕЯТЕЛЬНОСТИ СОЗДАНИЕ МОНОПОЛИЙ НА \\ 
\textbf{Режим 2}  & ФОРМИРОВАНИЕ ПОЛУЧЕНИЕ И РАСПРОСТРАНЕНИЕ \\ 
\textbf{Режим 2}  & Индикаторная буква: О \\ 
& ИНФОРМАЦИИ В РОССИЙСКОЙ ФЕДЕРАЦИИ \\ 
\textbf{Режим 4}  & В ТОМ ЧИСЛЕ С \\ 
\textbf{Режим 5}  & Пароль: ИНФОРМАЦИОННОЙ \\ 
& В ТОМ ЧИСЛЕ СО \\ 
	\end{tabular} 
\end{table}

\end{solution}
\begin{exercise}\begin{table}[H]
	\centering
	\begin{tabular}{r l}\textbf{Режим 1}  & ССФЭХЗУ ПЗЕЦЪЙАЪЭЪХП ДДЩЪАВВБ УОПЭОШЕТЖЧ \\ 
\textbf{Режим 2}  & ЕКВЕГЫПУМЩ ПАДЫП ЫВЛЩЩЬРРЩФСИНЩТ ЭЭГЪ \\ 
\textbf{Режим 3}  & Индикаторная буква: К \\ 
& ЛХ ТЬДАЛЦ К АЮФЧШХИР \\ 
\textbf{Режим 4}  & ЖыжнялыжМоэаижьжЖтэбиэыхВжензшшуЕт \\ 
\textbf{Режим 5}  & Пароль: СООБЩЕНИЙ \\ 
& МНРЩДЭЬЪСЕК С ДЗВП МИЕХЦЩЗШЫ \\ 
	\end{tabular} 
\end{table}

\end{exercise}
\begin{solution}
\begin{table}[H]
	\centering
	\begin{tabular}{r l}\textbf{Режим 1}  & СТОРОНЫ КРИМИНАЛЬНЫХ СТРУКТУР РЕАЛИЗАЦИИ \\ 
\textbf{Режим 2}  & ГРАЖДАНАМИ СВОИХ КОНСТИТУЦИОННЫХ ПРАВ \\ 
\textbf{Режим 2}  & Индикаторная буква: К \\ 
& НА ЛИЧНУЮ И СЕМЕЙНУЮ \\ 
\textbf{Режим 4}  & ТАЙНУ ТАЙНУ ПЕРЕПИСКИ ТЕЛЕФОННЫХ \\ 
\textbf{Режим 5}  & Пароль: СООБЩЕНИЙ \\ 
& ПЕРЕГОВОРОВ И ИНЫХ СООБЩЕНИЙ \\ 
	\end{tabular} 
\end{table}

\end{solution}
\begin{exercise}\begin{table}[H]
	\centering
	\begin{tabular}{r l}\textbf{Режим 1}  & НЖИДАЧОШРПАППР ХЦБЫЩАЪРРИ ЫКЫЮЯФВШНЗЧ ЦКЭЭЭЗЛ \\ 
\textbf{Режим 2}  & У ЛХЙФЫЫЮЗИНН ФПКЪЬЗЪМПЧД ГМЮЕМЩНПТА \\ 
\textbf{Режим 3}  & Индикаторная буква: М \\ 
& ИИФИЧЪЗЗЗЫЛЦЦЛ ЙЙЛЩАВСЛЯН РБКЩФПЭЪЭВА НФМДХХР \\ 
\textbf{Режим 4}  & ЛвюегъДшиьмюЩвзщпзПшкжкшФефызрЪкцмэьЗыыбхбЗэ \\ 
\textbf{Режим 5}  & Пароль: НЕИСПОЛНЕНИЕ \\ 
& Б ЭЬХЗВХЬЧЩГЭА ЬЪЛЬОЖБЧ ЩТШЕНЪГЬНЖБЧ \\ 
	\end{tabular} 
\end{table}

\end{exercise}
\begin{solution}
\begin{table}[H]
	\centering
	\begin{tabular}{r l}\textbf{Режим 1}  & НЕРАЦИОНАЛЬНОЕ ЧРЕЗМЕРНОЕ ОГРАНИЧЕНИЕ ДОСТУПА \\ 
\textbf{Режим 2}  & К ОБЩЕСТВЕННО НЕОБХОДИМОЙ ИНФОРМАЦИИ \\ 
\textbf{Режим 2}  & Индикаторная буква: М \\ 
& ПРОТИВОПРАВНОЕ ПРИМЕНЕНИЕ СПЕЦИАЛЬНЫХ СРЕДСТВ \\ 
\textbf{Режим 4}  & ВОЗДЕЙСТВИЯ НА ИНДИВИДУАЛЬНОЕ ГРУППОВОЕ \\ 
\textbf{Режим 5}  & Пароль: НЕИСПОЛНЕНИЕ \\ 
& И ОБЩЕСТВЕННОЕ СОЗНАНИЕ НЕИСПОЛНЕНИЕ \\ 
	\end{tabular} 
\end{table}

\end{solution}
\begin{exercise}\begin{table}[H]
	\centering
	\begin{tabular}{r l}\textbf{Режим 1}  & ФГВВЧАЪЭЪХВУ ВЩЭОУШЯЭ ШСАТЧЕНННЪЕЮЖЖЕ ЫЕАППЛ \\ 
\textbf{Режим 2}  & ЫЛХЮЯЧИХ ЧНЬСЙГКККВГПЦЦГ ТГЬММУ ЪДКПЭСЛАЮ \\ 
\textbf{Режим 3}  & Индикаторная буква: Ж \\ 
& УЯТАООЬЗКЖ ИЕЪЪШБКГЩ МЙАИЩЭЛА БНРРААШС \\ 
\textbf{Режим 4}  & УюуйпзчжЪъезцхжгЬдмаьълыЗздьштььДйндчцдкПпик \\ 
\textbf{Режим 5}  & Пароль: ИНФОРМАЦИОННОЙ \\ 
& ЪМРЫСЖМХН К БЩЪИЕВФЕЙНМИЪЯ ЕЦНСТ \\ 
	\end{tabular} 
\end{table}

\end{exercise}
\begin{solution}
\begin{table}[H]
	\centering
	\begin{tabular}{r l}\textbf{Режим 1}  & ФЕДЕРАЛЬНЫМИ ОРГАНАМИ ГОСУДАРСТВЕННОЙ ВЛАСТИ \\ 
\textbf{Режим 2}  & ОРГАНАМИ ГОСУДАРСТВЕННОЙ ВЛАСТИ СУБЪЕКТОВ \\ 
\textbf{Режим 2}  & Индикаторная буква: Ж \\ 
& РОССИЙСКОЙ ФЕДЕРАЦИИ ОРГАНАМИ МЕСТНОГО \\ 
\textbf{Режим 4}  & САМОУПРАВЛЕНИЯ ОРГАНИЗАЦИЯМИ И ГРАЖДАНАМИ \\ 
\textbf{Режим 5}  & Пароль: ИНФОРМАЦИОННОЙ \\ 
& ОТНОШЕНИЯ В ИНФОРМАЦИОННОЙ СФЕРЕ \\ 
	\end{tabular} 
\end{table}

\end{solution}
\begin{exercise}\begin{table}[H]
	\centering
	\begin{tabular}{r l}\textbf{Режим 1}  & НЖННЪДОЮДЗЧЧВ ЦБЦСЪЫЖЯРАИ ЭБЛЛЛАП ЖСМЖЦЪО \\ 
\textbf{Режим 2}  & У ЛЩЯЫЙЫЕИ ХНТЭНОГСЦШЖЧСЙ ЧЛПЧММНР \\ 
\textbf{Режим 3}  & Индикаторная буква: В \\ 
& НДГГЮУЕИЕФЖ ЪДЬФЛЛК КУЫЛЗПЯЯЯЙПХООП ЩПЫШШГ \\ 
\textbf{Режим 4}  & ЩокдщТйцбыЖаблоУужюшДмпккЩочъюКкгъаТжшунДпзьфУы \\ 
\textbf{Режим 5}  & Пароль: МЕСТНОГО \\ 
& ЬЭЦЖБЯЭАНЗ ЖРЯНФОБШР ЦДФГБНЙ ЪРЕЩОЫПЭ \\ 
	\end{tabular} 
\end{table}

\end{exercise}
\begin{solution}
\begin{table}[H]
	\centering
	\begin{tabular}{r l}\textbf{Режим 1}  & НЕПРАВОМЕРНОЕ ОГРАНИЧЕНИЕ ДОСТУПА ГРАЖДАН \\ 
\textbf{Режим 2}  & К ОТКРЫТЫМ ИНФОРМАЦИОННЫМ РЕСУРСАМ \\ 
\textbf{Режим 2}  & Индикаторная буква: В \\ 
& ФЕДЕРАЛЬНЫХ ОРГАНОВ ГОСУДАРСТВЕННОЙ ВЛАСТИ \\ 
\textbf{Режим 4}  & ОРГАНОВ ГОСУДАРСТВЕННОЙ ВЛАСТИ СУБЪЕКТОВ \\ 
\textbf{Режим 5}  & Пароль: МЕСТНОГО \\ 
& РОССИЙСКОЙ ФЕДЕРАЦИИ ОРГАНОВ МЕСТНОГО \\ 
	\end{tabular} 
\end{table}

\end{solution}
\begin{exercise}\begin{table}[H]
	\centering
	\begin{tabular}{r l}\textbf{Режим 1}  & СЕХКИХХРВЦЩЦГЯ Г ЕЦРДЗДЮБ ОЛГНШСДН \\ 
\textbf{Режим 2}  & ФЕЬЦФПВЧЛП М РЧЮУЕР ЪМЫГЮГЫС \\ 
\textbf{Режим 3}  & Индикаторная буква: Ш \\ 
& ЛТЩФПЭЪЭЭ ЖХГЯЦЖЮМ ДЦЗЙЦЪЬЖБУ ЬЬСЫЛХЮЯФХЦЫЗД \\ 
\textbf{Режим 4}  & ЙосйзсЗщмэуьЮкынылНавощнВчзожеРюйл \\ 
\textbf{Режим 5}  & Пароль: ЦЕННОСТЕЙ \\ 
& Х ЬГПРСЙЧНХД ЛЙТЭЮЯСИВП ЧНДЙЭЯТЧЫ \\ 
	\end{tabular} 
\end{table}

\end{exercise}
\begin{solution}
\begin{table}[H]
	\centering
	\begin{tabular}{r l}\textbf{Режим 1}  & САМОУПРАВЛЕНИЯ К ОТКРЫТЫМ АРХИВНЫМ \\ 
\textbf{Режим 2}  & МАТЕРИАЛАМ К ДРУГОЙ ОТКРЫТОЙ \\ 
\textbf{Режим 2}  & Индикаторная буква: Ш \\ 
& СОЦИАЛЬНО ЗНАЧИМОЙ ИНФОРМАЦИИ ДЕЗОРГАНИЗАЦИЯ \\ 
\textbf{Режим 4}  & И РАЗРУШЕНИЕ СИСТЕМЫ НАКОПЛЕНИЯ \\ 
\textbf{Режим 5}  & Пароль: ЦЕННОСТЕЙ \\ 
& И СОХРАНЕНИЯ КУЛЬТУРНЫХ ЦЕННОСТЕЙ \\ 
	\end{tabular} 
\end{table}

\end{solution}
\begin{exercise}\begin{table}[H]
	\centering
	\begin{tabular}{r l}\textbf{Режим 1}  & ВЙЙАКАС СМЧБИШ ЛХРЩЕКЯФЗ ФИЭЬЬЖННЬЙЮЧПЬО \\ 
\textbf{Режим 2}  & ЯЯЙЦ Ш ИЕХГЗЪ ИГЙПБЛГС \\ 
\textbf{Режим 3}  & Индикаторная буква: Ф \\ 
& Т НВОНКЮЯФЬЦ П НМОЪЭЭЦ \\ 
\textbf{Режим 4}  & ДхдиифмКрталерШыашъииЫряогзчВешцехзОввсяваАщз \\ 
\textbf{Режим 5}  & Пароль: МАССОВОЙ \\ 
& ЗМЖЕРЮОЫЗЙДДЧИ ОЧЖМПЦЩГ ВСЖЫЦПК ЮОЯСЙАЕЯ \\ 
	\end{tabular} 
\end{table}

\end{exercise}
\begin{solution}
\begin{table}[H]
	\centering
	\begin{tabular}{r l}\textbf{Режим 1}  & ВКЛЮЧАЯ АРХИВЫ НАРУШЕНИЕ КОНСТИТУЦИОННЫХ \\ 
\textbf{Режим 2}  & ПРАВ И СВОБОД ЧЕЛОВЕКА \\ 
\textbf{Режим 2}  & Индикаторная буква: Ф \\ 
& И ГРАЖДАНИНА В ОБЛАСТИ \\ 
\textbf{Режим 4}  & МАССОВОЙ ИНФОРМАЦИИ ВЫТЕСНЕНИЕ РОССИЙСКИХ \\ 
\textbf{Режим 5}  & Пароль: МАССОВОЙ \\ 
& ИНФОРМАЦИОННЫХ АГЕНТСТВ СРЕДСТВ МАССОВОЙ \\ 
	\end{tabular} 
\end{table}

\end{solution}
\begin{exercise}\begin{table}[H]
	\centering
	\begin{tabular}{r l}\textbf{Режим 1}  & ИАРЯАНЙБШЮ Н ДЗЭКЗГПЦРБМ РЪПГЪЛКМСБТААШС \\ 
\textbf{Режим 2}  & ОАЧМУ Г ЮПРЩТЩУС ЯХКЬЛНСЯТТО \\ 
\textbf{Режим 3}  & Индикаторная буква: Ю \\ 
& ХЛРСЧННЗ ЩШХЗЗЖМНЩЮДЙЩ Р ЕЪРУДАЙНРСТН \\ 
\textbf{Режим 4}  & ИмптеъьзЧоубюоссЖанкэфяэПастткк \\ 
\textbf{Режим 5}  & Пароль: СТРУКТУР \\ 
& ЕЮ МУЩШБДЫЙТХ АЙОШЙМРКАЦЦЮФО ХШЩШЕКЧЧ \\ 
	\end{tabular} 
\end{table}

\end{exercise}
\begin{solution}
\begin{table}[H]
	\centering
	\begin{tabular}{r l}\textbf{Режим 1}  & ИНФОРМАЦИИ С ВНУТРЕННЕГО ИНФОРМАЦИОННОГО \\ 
\textbf{Режим 2}  & РЫНКА И УСИЛЕНИЕ ЗАВИСИМОСТИ \\ 
\textbf{Режим 2}  & Индикаторная буква: Ю \\ 
& ДУХОВНОЙ ЭКОНОМИЧЕСКОЙ И ПОЛИТИЧЕСКОЙ \\ 
\textbf{Режим 4}  & СФЕР ОБЩЕСТВЕННОЙ ЖИЗНИ РОССИИ \\ 
\textbf{Режим 5}  & Пароль: СТРУКТУР \\ 
& ОТ ЗАРУБЕЖНЫХ ИНФОРМАЦИОННЫХ СТРУКТУР \\ 
	\end{tabular} 
\end{table}

\end{solution}
\begin{exercise}\begin{table}[H]
	\centering
	\begin{tabular}{r l}\textbf{Режим 1}  & ДДВЫЙСБАЗРИ АЙЦВЭЫЧЪ ЪКЯССААЖЗ ЭЭХХВГЛЧВУ \\ 
\textbf{Режим 2}  & ЫОЫЮФВСЧ ЭМЬИКДОП ПКЦЯШШПЭ ЕЙГГИФЛЫЧЪ \\ 
\textbf{Режим 3}  & Индикаторная буква: П \\ 
& ЬЦ КАХВИЙ ЗВХМЦДТ ЙА \\ 
\textbf{Режим 4}  & ШквгуЬяъкжМзэьмЦчатчЕчффвЯшзыеХылщгЪгя \\ 
\textbf{Режим 5}  & Пароль: ПРИНЯТЫМ \\ 
& ЯОЪМВБЖЬЗЬИШВО ЭЖЬЮЪЕЧАП ИАЙГЭЧИП Ц \\ 
	\end{tabular} 
\end{table}

\end{exercise}
\begin{solution}
\begin{table}[H]
	\centering
	\begin{tabular}{r l}\textbf{Режим 1}  & ДЕВАЛЬВАЦИЯ ДУХОВНЫХ ЦЕННОСТЕЙ ПРОПАГАНДА \\ 
\textbf{Режим 2}  & ОБРАЗЦОВ МАССОВОЙ КУЛЬТУРЫ ОСНОВАННЫХ \\ 
\textbf{Режим 2}  & Индикаторная буква: П \\ 
& НА КУЛЬТЕ НАСИЛИЯ НА \\ 
\textbf{Режим 4}  & ДУХОВНЫХ И НРАВСТВЕННЫХ ЦЕННОСТЯХ \\ 
\textbf{Режим 5}  & Пароль: ПРИНЯТЫМ \\ 
& ПРОТИВОРЕЧАЩИХ ЦЕННОСТЯМ ПРИНЯТЫМ В \\ 
	\end{tabular} 
\end{table}

\end{solution}
\begin{exercise}\begin{table}[H]
	\centering
	\begin{tabular}{r l}\textbf{Режим 1}  & РБЛЫЭЭТФИФ ЭЕБЙФФЩЪ ЗПЪВЛЕРУ АЙЦВЭЫЫКУ \\ 
\textbf{Режим 2}  & ЩЪФЭЩЩХФАЯЯЮЬ О ЯЙФЭАМКПЮВЧ ЧЦЖЫДЮЫНЛФ \\ 
\textbf{Режим 3}  & Индикаторная буква: Л \\ 
& ЖЫШВЕЛЕРИ ЕГДЪАЯ МВУ ЫЛЪЗААЦЧКХХ \\ 
\textbf{Режим 4}  & ОывтыКслаъЧаязшЗбвбхПэбуаЙтжипХжевиШлвылОык \\ 
\textbf{Режим 5}  & Пароль: ИСПОЛЬЗОВАНИЯ \\ 
& АГЦ КЖУФРЕМБН В ЫЦЯЫПФИЫГАЩЫС \\ 
	\end{tabular} 
\end{table}

\end{exercise}
\begin{solution}
\begin{table}[H]
	\centering
	\begin{tabular}{r l}\textbf{Режим 1}  & РОССИЙСКОМ ОБЩЕСТВЕ СНИЖЕНИЕ ДУХОВНОГО \\ 
\textbf{Режим 2}  & НРАВСТВЕННОГО И ТВОРЧЕСКОГО ПОТЕНЦИАЛА \\ 
\textbf{Режим 2}  & Индикаторная буква: Л \\ 
& НАСЕЛЕНИЯ РОССИИ ЧТО СУЩЕСТВЕННО \\ 
\textbf{Режим 4}  & ОСЛОЖНИТ ПОДГОТОВКУ ТРУДОВЫХ РЕСУРСОВ \\ 
\textbf{Режим 5}  & Пароль: ИСПОЛЬЗОВАНИЯ \\ 
& ДЛЯ ВНЕДРЕНИЯ И ИСПОЛЬЗОВАНИЯ \\ 
	\end{tabular} 
\end{table}

\end{solution}
\begin{exercise}\begin{table}[H]
	\centering
	\begin{tabular}{r l}\textbf{Режим 1}  & ННЙПЮТПИ КЪФППЕЙУРР С МЩЛ \\ 
\textbf{Режим 2}  & ЩКАФЖ ШКСХКЮРНЕЧЦЙНШ ЪЬГБГЕБЯРБОЮЯФЗ ХНТЭНОГСЦДЙ \\ 
\textbf{Режим 3}  & Индикаторная буква: Б \\ 
& (ВВЩЩЕЖДЕВЭЦАО ВБСБМБКШ ОКШ ЮНЮЩМДЖЕЩ \\ 
\textbf{Режим 4}  & ЬлъчдИещижЗььшшВэъхзАваещОсбъыОруоеЦхьииЭщрмщЯаыпйЮкъкаЛмб \\ 
\textbf{Режим 5}  & Пароль: ФЕДЕРАЦИИ \\ 
& ИЭИЯЪАЯМЙНЧКФШГ СЪВТСЮОЙ ПЬММТЗИОШГ БТАСИДУЙЩ \\ 
	\end{tabular} 
\end{table}

\end{exercise}
\begin{solution}
\begin{table}[H]
	\centering
	\begin{tabular}{r l}\textbf{Режим 1}  & НОВЕЙШИХ ТЕХНОЛОГИЙ В ТОМ \\ 
\textbf{Режим 2}  & ЧИСЛЕ ИНФОРМАЦИОННЫХ МАНИПУЛИРОВАНИЕ ИНФОРМАЦИЕЙ \\ 
\textbf{Режим 2}  & Индикаторная буква: Б \\ 
& (ДЕЗИНФОРМАЦИЯ СОКРЫТИЕ ИЛИ ИСКАЖЕНИЕ \\ 
\textbf{Режим 4}  & ИНФОРМАЦИИ). УГРОЗАМИ ИНФОРМАЦИОННОМУ ОБЕСПЕЧЕНИЮ \\ 
\textbf{Режим 5}  & Пароль: ФЕДЕРАЦИИ \\ 
& ГОСУДАРСТВЕННОЙ ПОЛИТИКИ РОССИЙСКОЙ ФЕДЕРАЦИИ \\ 
	\end{tabular} 
\end{table}

\end{solution}
\begin{exercise}\begin{table}[H]
	\centering
	\begin{tabular}{r l}\textbf{Режим 1}  & МЙУЧЦ НЬЛКЪЮВШ АТЯЯЯЬФЗЮДАЧВ ЦЮФЖЮЙТОВЕЪГГЬВ \\ 
\textbf{Режим 2}  & ОАЧМУ ЧЙПЦЛЫ АЬВ РГЭЭЯЦЯЕЛ \\ 
\textbf{Режим 3}  & Индикаторная буква: Б \\ 
& ШВДЖМЙДС ГЕСЦИВВОКЯСДНО З ЮДЭИГЕЕЖЯЦЪ \\ 
\textbf{Режим 4}  & ЫвжзхщыЪэниецхЖямвбухЯбпчтдяЦааьчькСрджейъФфцтт \\ 
\textbf{Режим 5}  & Пароль: РОССИЙСКОЙ \\ 
& ЗЫ РДПИЙЫССШКСДЫЕ ЙРВЯЙЯЦЩЪХ Р \\ 
	\end{tabular} 
\end{table}

\end{exercise}
\begin{solution}
\begin{table}[H]
	\centering
	\begin{tabular}{r l}\textbf{Режим 1}  & МОГУТ ЯВЛЯТЬСЯ МОНОПОЛИЗАЦИЯ ИНФОРМАЦИОННОГО \\ 
\textbf{Режим 2}  & РЫНКА РОССИИ ЕГО ОТДЕЛЬНЫХ \\ 
\textbf{Режим 2}  & Индикаторная буква: Б \\ 
& СЕКТОРОВ ОТЕЧЕСТВЕННЫМИ И ЗАРУБЕЖНЫМИ \\ 
\textbf{Режим 4}  & ГОСУДАРСТВЕННЫХ СРЕДСТВ МАССОВОЙ ИНФОРМАЦИИ \\ 
\textbf{Режим 5}  & Пароль: РОССИЙСКОЙ \\ 
& ПО ИНФОРМИРОВАНИЮ РОССИЙСКОЙ И \\ 
	\end{tabular} 
\end{table}

\end{solution}
\begin{exercise}\begin{table}[H]
	\centering
	\begin{tabular}{r l}\textbf{Режим 1}  & ЗМИСДЙЙШШЙ ЫОРГЖМЙЛЫ ЩУАУХЗ ЖЩВХНУХПЭЭИИЛ \\ 
\textbf{Режим 2}  & ОЮЗШЪЪЧГЕР ПАССЩХЕЬИ ЗБИЗЮЯЯЙЮЦ ММНЙНЕЗУ \\ 
\textbf{Режим 3}  & Индикаторная буква: Е \\ 
& ОМЪШПНЙНЕЮПБАЕДКЧ ЛЭЬЩЛК БРЫЛБТТПЗД ИЖЭЭЕЖА \\ 
\textbf{Режим 4}  & БхчюцЙосипЛлжмыФтгсфЯъвряПрккюЕэьяпИиемйЩъмззЬдр \\ 
\textbf{Режим 5}  & Пароль: РАЗВИТИЮ \\ 
& ЙМГЗЕМИГЙЙУШЪЪ ДТЧЩВЦРГ. ЦООЙИВЩГ ЕЮМУЬФЫЯ \\ 
	\end{tabular} 
\end{table}

\end{exercise}
\begin{solution}
\begin{table}[H]
	\centering
	\begin{tabular}{r l}\textbf{Режим 1}  & ЗАРУБЕЖНОЙ АУДИТОРИИ НИЗКАЯ ЭФФЕКТИВНОСТЬ \\ 
\textbf{Режим 2}  & РОССИЙСКОЙ ФЕДЕРАЦИИ ВСЛЕДСТВИЕ ДЕФИЦИТА \\ 
\textbf{Режим 2}  & Индикаторная буква: Е \\ 
& КВАЛИФИЦИРОВАННЫХ КАДРОВ ОТСУТСТВИЯ СИСТЕМЫ \\ 
\textbf{Режим 4}  & ФОРМИРОВАНИЯ И РЕАЛИЗАЦИИ ГОСУДАРСТВЕННОЙ \\ 
\textbf{Режим 5}  & Пароль: РАЗВИТИЮ \\ 
& ИНФОРМАЦИОННОЙ ПОЛИТИКИ. УГРОЗАМИ РАЗВИТИЮ \\ 
	\end{tabular} 
\end{table}

\end{solution}
\begin{exercise}\begin{table}[H]
	\centering
	\begin{tabular}{r l}\textbf{Режим 1}  & ОФДИГШШБЛЕДДЛ БВЬДЩЩЛИН ЭСЫЬСФЕТЖЧ ЪЧЧБНБТ \\ 
\textbf{Режим 2}  & ЫЩСЕГГРЬЖ ЫУОШААЦ ШКСХКЮРХЕЪУХЩВ ПУВАЫВЫБГУИЬЖВКХ \\ 
\textbf{Режим 3}  & Индикаторная буква: Л \\ 
& М ЗЛТЩЩ МЯУМЪЯПЬЫСЮ БТЫТЗПННЬЬЦЖ \\ 
\textbf{Режим 4}  & ЬяъпцмуъМэжпньщэИристт \\ 
\textbf{Режим 5}  & Пароль: ПРОДУКЦИИ \\ 
& ЯОЫЦЗУЛЫЫ К ЩЧАПИУ ФЙСЦ \\ 
	\end{tabular} 
\end{table}

\end{exercise}
\begin{solution}
\begin{table}[H]
	\centering
	\begin{tabular}{r l}\textbf{Режим 1}  & ОТЕЧЕСТВЕННОЙ ИНДУСТРИИ ИНФОРМАЦИИ ВКЛЮЧАЯ \\ 
\textbf{Режим 2}  & ИНДУСТРИЮ СРЕДСТВ ИНФОРМАТИЗАЦИИ ТЕЛЕКОММУНИКАЦИИ \\ 
\textbf{Режим 2}  & Индикаторная буква: Л \\ 
& И СВЯЗИ ОБЕСПЕЧЕНИЮ ПОТРЕБНОСТЕЙ \\ 
\textbf{Режим 4}  & ВНУТРЕННЕГО РЫНКА В ЕЕ \\ 
\textbf{Режим 5}  & Пароль: ПРОДУКЦИИ \\ 
& ПРОДУКЦИИ И ВЫХОДУ ЭТОЙ \\ 
	\end{tabular} 
\end{table}

\end{solution}
\begin{exercise}\begin{table}[H]
	\centering
	\begin{tabular}{r l}\textbf{Режим 1}  & ППМУПРЧУТ ЛХ УЬЫТШЯФ ЯГЭЭЗ \\ 
\textbf{Режим 2}  & А ЧСВУА ЩЭЬЩЛЭДКЯФЙ ИЙХЭЭЗЙЗЦЛ \\ 
\textbf{Режим 3}  & Индикаторная буква: Ъ \\ 
& НХЬХРЖЧЧЮЮЩ В НЧПАЫДАФББХТ ЭУЯЬФГЮФЕГЗЦЛ \\ 
\textbf{Режим 4}  & ИъйтютмИйстггхУхъцопжКыкйарлЩзбькмфНйдегьвЖазлы \\ 
\textbf{Режим 5}  & Пароль: РОССИЙСКОЙ \\ 
& ЛКГЬЙЪЦХ ЛСШЩРАЪЖБТВАЩСБ УЪЯПДЛТ ОЫЦЦЫХЦУЫХ \\ 
	\end{tabular} 
\end{table}

\end{exercise}
\begin{solution}
\begin{table}[H]
	\centering
	\begin{tabular}{r l}\textbf{Режим 1}  & ПРОДУКЦИИ НА МИРОВОЙ РЫНОК \\ 
\textbf{Режим 2}  & А ТАКЖЕ ОБЕСПЕЧЕНИЮ НАКОПЛЕНИЯ \\ 
\textbf{Режим 2}  & Индикаторная буква: Ъ \\ 
& СОХРАННОСТИ И ЭФФЕКТИВНОГО ИСПОЛЬЗОВАНИЯ \\ 
\textbf{Режим 4}  & ОТЕЧЕСТВЕННЫХ ИНФОРМАЦИОННЫХ РЕСУРСОВ МОГУТ \\ 
\textbf{Режим 5}  & Пароль: РОССИЙСКОЙ \\ 
& ЯВЛЯТЬСЯ ПРОТИВОДЕЙСТВИЕ ДОСТУПУ РОССИЙСКОЙ \\ 
	\end{tabular} 
\end{table}

\end{solution}
\begin{exercise}\begin{table}[H]
	\centering
	\begin{tabular}{r l}\textbf{Режим 1}  & ФГВВЧАЗРЛ Р ЩЩНЬАЙЬА НЯЛСЯЭМУЮКОЗУЖ \\ 
\textbf{Режим 2}  & ЮЛШДДВГЬТХБ КИШНСЯЖЪЧНПККЗН Ц ОЛРППППЯАЪЪВМ \\ 
\textbf{Режим 3}  & Индикаторная буква: И \\ 
& ЦЖЭЪЪЯШ ЛУЫБФФАКЗТ ЭЭХПЙЩЮЩДШРДЫВ Я \\ 
\textbf{Режим 4}  & ЙэосипРшжормЮхкскаЖээыулЦюцч \\ 
\textbf{Режим 5}  & Пароль: СРЕДСТВ \\ 
& РЮПСЦЮХРЙ ШКЖЦХЪРУЮЕЙШФН ЯИГШЪ ЦОЧЦЦЮГ \\ 
	\end{tabular} 
\end{table}

\end{exercise}
\begin{solution}
\begin{table}[H]
	\centering
	\begin{tabular}{r l}\textbf{Режим 1}  & ФЕДЕРАЦИИ К НОВЕЙШИМ ИНФОРМАЦИОННЫМ \\ 
\textbf{Режим 2}  & ТЕХНОЛОГИЯМ ВЗАИМОВЫГОДНОМУ И РАВНОПРАВНОМУ \\ 
\textbf{Режим 2}  & Индикаторная буква: И \\ 
& УЧАСТИЮ РОССИЙСКИХ ПРОИЗВОДИТЕЛЕЙ В \\ 
\textbf{Режим 4}  & МИРОВОМ РАЗДЕЛЕНИИ ТРУДА В \\ 
\textbf{Режим 5}  & Пароль: СРЕДСТВ \\ 
& ИНДУСТРИИ ИНФОРМАЦИОННЫХ УСЛУГ СРЕДСТВ \\ 
	\end{tabular} 
\end{table}

\end{solution}
\begin{exercise}\begin{table}[H]
	\centering
	\begin{tabular}{r l}\textbf{Режим 1}  & ИАРЯАНЙЯШЖЪЯПЦ КЪЦЩЕЦЕДЧЪЫЛЭЦАЯ С РОЖФФ \\ 
\textbf{Режим 2}  & ЫЩЙВЩБОДИТЯСДЫ ЬЬЭЦЬЮЭЗЩ Л ОБЕВЛ \\ 
\textbf{Режим 3}  & Индикаторная буква: И \\ 
& МГТЬКЫСЭ ВЫЬЯЖХХ ЧХЪ ЬЭПЖДЖЕУ \\ 
\textbf{Режим 4}  & ЛхъэжюпМнпзыжсЙкихзйпАщчщейчИйыеъммЧффю \\ 
\textbf{Режим 5}  & Пароль: ТЕХНОЛОГИЙ \\ 
& ЦМУНВХС ЭЪПЧЧБАБЖЧС ЫНЗЭВВОНСХГНИТ ГАДЖЫШЪЧГЗ \\ 
	\end{tabular} 
\end{table}

\end{exercise}
\begin{solution}
\begin{table}[H]
	\centering
	\begin{tabular}{r l}\textbf{Режим 1}  & ИНФОРМАТИЗАЦИИ ТЕЛЕКОММУНИКАЦИИ И СВЯЗИ \\ 
\textbf{Режим 2}  & ИНФОРМАЦИОННЫХ ПРОДУКТОВ А ТАКЖЕ \\ 
\textbf{Режим 2}  & Индикаторная буква: И \\ 
& СОЗДАНИЕ УСЛОВИЙ ДЛЯ УСИЛЕНИЯ \\ 
\textbf{Режим 4}  & ТЕХНОЛОГИЧЕСКОЙ ЗАВИСИМОСТИ РОССИИ В \\ 
\textbf{Режим 5}  & Пароль: ТЕХНОЛОГИЙ \\ 
& ОБЛАСТИ СОВРЕМЕННЫХ ИНФОРМАЦИОННЫХ ТЕХНОЛОГИЙ \\ 
	\end{tabular} 
\end{table}

\end{solution}
\begin{exercise}\begin{table}[H]
	\centering
	\begin{tabular}{r l}\textbf{Режим 1}  & ЗМОСФЮЩ ЗОЩЫПУМЩ УЕЙЦЯЭГГГКЭТААЭ ЖЭЙССЮ \\ 
\textbf{Режим 2}  & ЫВГВЩГБПГ ЫУОШААЦ ШКСХКЮРХЕЪУХЩВ ПУВАЫВЫБГУИЬЖВКХ \\ 
\textbf{Режим 3}  & Индикаторная буква: Ч \\ 
& Ф ТЧМШШ ЭЭЧ ЗВЧМНДЪ \\ 
\textbf{Режим 4}  & КръьцьгХгфиыыжЛэлжлпюЭьщивямШлрвбшжВаци \\ 
\textbf{Режим 5}  & Пароль: ЗАРУБЕЖНЫМ \\ 
& ЫЙ ХЫЧШС ПЫЬЗЪКВЮШБМЕХЗЕ МУЮЯЮАЪЭПЕ \\ 
	\end{tabular} 
\end{table}

\end{exercise}
\begin{solution}
\begin{table}[H]
	\centering
	\begin{tabular}{r l}\textbf{Режим 1}  & ЗАКУПКА ОРГАНАМИ ГОСУДАРСТВЕННОЙ ВЛАСТИ \\ 
\textbf{Режим 2}  & ИМПОРТНЫХ СРЕДСТВ ИНФОРМАТИЗАЦИИ ТЕЛЕКОММУНИКАЦИИ \\ 
\textbf{Режим 2}  & Индикаторная буква: Ч \\ 
& И СВЯЗИ ПРИ НАЛИЧИИ \\ 
\textbf{Режим 4}  & ОТЕЧЕСТВЕННЫХ АНАЛОГОВ НЕ УСТУПАЮЩИХ \\ 
\textbf{Режим 5}  & Пароль: ЗАРУБЕЖНЫМ \\ 
& ПО СВОИМ ХАРАКТЕРИСТИКАМ ЗАРУБЕЖНЫМ \\ 
	\end{tabular} 
\end{table}

\end{solution}
\begin{exercise}\begin{table}[H]
	\centering
	\begin{tabular}{r l}\textbf{Режим 1}  & ОЩОЛДФУМ УЭПУМЩЯРАИ В БРКГОТТПЖФККЕО \\ 
\textbf{Режим 2}  & ОАЧМУ ЧЙПЦЛЛЕЫТМ РРБИЭШЯШХАЖХПЮ ЭКЕЪООЛ \\ 
\textbf{Режим 3}  & Индикаторная буква: Б \\ 
& ДЦЗЙЦЪЬЙБТФЙЯС ВФСХНСНЭАФЮШГСЦЙ М ЗЛТЩЩ \\ 
\textbf{Режим 4}  & ЗщхэаГчайоСрбепБзчмбЗизлщМйжю \\ 
\textbf{Режим 5}  & Пароль: ИНТЕЛЛЕКТУАЛЬНОЙ \\ 
& МЪРУМЛХАЖШЙВ Л ЪРЙСГЦМПЛПКЮВДЪР ББАТРЕЧЦХЯКЕСМЮР \\ 
	\end{tabular} 
\end{table}

\end{exercise}
\begin{solution}
\begin{table}[H]
	\centering
	\begin{tabular}{r l}\textbf{Режим 1}  & ОБРАЗЦАМ ВЫТЕСНЕНИЕ С ОТЕЧЕСТВЕННОГО \\ 
\textbf{Режим 2}  & РЫНКА РОССИЙСКИХ ПРОИЗВОДИТЕЛЕЙ СРЕДСТВ \\ 
\textbf{Режим 2}  & Индикаторная буква: Б \\ 
& ИНФОРМАТИЗАЦИИ ТЕЛЕКОММУНИКАЦИИ И СВЯЗИ \\ 
\textbf{Режим 4}  & УВЕЛИЧЕНИЕ ОТТОКА ЗА РУБЕЖ \\ 
\textbf{Режим 5}  & Пароль: ИНТЕЛЛЕКТУАЛЬНОЙ \\ 
& СПЕЦИАЛИСТОВ И ПРАВООБЛАДАТЕЛЕЙ ИНТЕЛЛЕКТУАЛЬНОЙ \\ 
	\end{tabular} 
\end{table}

\end{solution}
\begin{exercise}\begin{table}[H]
	\centering
	\begin{tabular}{r l}\textbf{Режим 1}  & И ЛХИЗФИФКСХЧЖВИМЖПЦИЗ ПЙУАЕЕН А \\ 
\textbf{Режим 2}  & ЦЛММЯЛ УХА ЩЭК УЧХЧЮЬФСИЫЬ \\ 
\textbf{Режим 3}  & Индикаторная буква: Х \\ 
& ГОЯ И ЫАКШЦПМЧХЛС ЗВ \\ 
\textbf{Режим 4}  & ДьпннЦхтьяАщщпйЭддууШаухвРклщчЮжлцыЫб \\ 
\textbf{Режим 5}  & Пароль: ИСПОЛЬЗОВАНИЕ \\ 
& ДЙСЩМЯБЛХАЬГВТ ЦЧЫЗ Л ЬВОЙРКЮЪАПБМУ \\ 
	\end{tabular} 
\end{table}

\end{exercise}
\begin{solution}
\begin{table}[H]
	\centering
	\begin{tabular}{r l}\textbf{Режим 1}  & И ТЕЛЕКОММУНИКАЦИОННЫХ СРЕДСТВ И \\ 
\textbf{Режим 2}  & СИСТЕМ КАК УЖЕ РАЗВЕРНУТЫХ \\ 
\textbf{Режим 2}  & Индикаторная буква: Х \\ 
& ТАК И СОЗДАВАЕМЫХ НА \\ 
\textbf{Режим 4}  & ТЕРРИТОРИИ РОССИИ МОГУТ ЯВЛЯТЬСЯ \\ 
\textbf{Режим 5}  & Пароль: ИСПОЛЬЗОВАНИЕ \\ 
& ПРОТИВОПРАВНЫЕ СБОР И ИСПОЛЬЗОВАНИЕ \\ 
	\end{tabular} 
\end{table}

\end{solution}
\begin{exercise}\begin{table}[H]
	\centering
	\begin{tabular}{r l}\textbf{Режим 1}  & ИАРЯАНЙБШЮ ХГКЭСЪЧДТ ЮЛШДДВГЬТА ЛХЛШШАБЭХ \\ 
\textbf{Режим 2}  & ЫЩЙВЩБОДИН ШСЗЮСЖФЮЦ Д ЩОЗЛШБШЙНЫ \\ 
\textbf{Режим 3}  & Индикаторная буква: К \\ 
& А РРБХБЖЬИИГЖ ШЖЙЙЮЦЛ ЦЖЦЧЦМЫДЦЪЬ \\ 
\textbf{Режим 4}  & ЬмуьщлЦоумщхБкхопмЪажжхцМиьжчаЪнокиэТрййир \\ 
\textbf{Режим 5}  & Пароль: ИЗДЕЛИЯ \\ 
& АБЗЯЧТЕЙЗТШЪЛ ЕИ ЖЬШ МББТАЮД \\ 
	\end{tabular} 
\end{table}

\end{exercise}
\begin{solution}
\begin{table}[H]
	\centering
	\begin{tabular}{r l}\textbf{Режим 1}  & ИНФОРМАЦИИ НАРУШЕНИЯ ТЕХНОЛОГИИ ОБРАБОТКИ \\ 
\textbf{Режим 2}  & ИНФОРМАЦИИ ВНЕДРЕНИЕ В АППАРАТНЫЕ \\ 
\textbf{Режим 2}  & Индикаторная буква: К \\ 
& И ПРОГРАММНЫЕ ИЗДЕЛИЯ КОМПОНЕНТОВ \\ 
\textbf{Режим 4}  & РЕАЛИЗУЮЩИХ ФУНКЦИИ НЕ ПРЕДУСМОТРЕННЫЕ \\ 
\textbf{Режим 5}  & Пароль: ИЗДЕЛИЯ \\ 
& ДОКУМЕНТАЦИЕЙ НА ЭТИ ИЗДЕЛИЯ \\ 
	\end{tabular} 
\end{table}

\end{solution}
\begin{exercise}\begin{table}[H]
	\centering
	\begin{tabular}{r l}\textbf{Режим 1}  & РЗНБЖЖСТКМ Ш НЪЭХХШООЮУЙЛЕРУ ДДЩИЩХУТ \\ 
\textbf{Режим 2}  & Ы БВМРВУЗЪНАСЬЬ-АЖХПШХШЮФЖЕМУХЩМДЪХП ДТГГНТ О \\ 
\textbf{Режим 3}  & Индикаторная буква: М \\ 
& ЯЭХ ТЧФЖД ОЕШШРЕ РСОЫМЗ \\ 
\textbf{Режим 4}  & НъгянУымвцЗьэьаЛмзлеФгечсИгытмЮбкхыГнд \\ 
\textbf{Режим 5}  & Пароль: ИНФОРМАЦИИ \\ 
& Ы ЕТВКЖЕ ЬЬЕИИВЩПЗ ЩИПЪЕВФЕЙЗ \\ 
	\end{tabular} 
\end{table}

\end{exercise}
\begin{solution}
\begin{table}[H]
	\centering
	\begin{tabular}{r l}\textbf{Режим 1}  & РАЗРАБОТКА И РАСПРОСТРАНЕНИЕ ПРОГРАММ \\ 
\textbf{Режим 2}  & И ИНФОРМАЦИОННО-ТЕЛЕКОММУНИКАЦИОННЫХ СИСТЕМ В \\ 
\textbf{Режим 2}  & Индикаторная буква: М \\ 
& ТОМ ЧИСЛЕ СИСТЕМ ЗАЩИТЫ \\ 
\textbf{Режим 4}  & ПОДАВЛЕНИЕ ИЛИ РАЗРУШЕНИЕ СРЕДСТВ \\ 
\textbf{Режим 5}  & Пароль: ИНФОРМАЦИИ \\ 
& И СИСТЕМ ОБРАБОТКИ ИНФОРМАЦИИ \\ 
	\end{tabular} 
\end{table}

\end{solution}
\begin{exercise}\begin{table}[H]
	\centering
	\begin{tabular}{r l}\textbf{Режим 1}  & ТЮКЧЗКЗШЭЮМЙБКГЩ В ЦЯНУУ НРСЭЭЬЫЫЮКШ \\ 
\textbf{Режим 2}  & ЩЭ ЩКВЫАПАА-ННМЛЖЙРЦ ЭПККЪПС ГТФВПК \\ 
\textbf{Режим 3}  & Индикаторная буква: Я \\ 
& ЯАЦЪВЭДАССЛУЗЖСЬЪБ СШИИЙШ ЗВЗЫЫПЮЪВ Р \\ 
\textbf{Режим 4}  & ЯйыцыбяЬшллъчдЬмгьюллЕоэкниэЬгуцьюлЭфылкчяЧэ \\ 
\textbf{Режим 5}  & Пароль: КРИПТОГРАФИЧЕСКОЙ \\ 
& А ХЕЗЛВНЩ ЪГЫАСЕНРЙЕЙЮЗЖАХЦ ВФЗФСФ \\ 
	\end{tabular} 
\end{table}

\end{exercise}
\begin{solution}
\begin{table}[H]
	\centering
	\begin{tabular}{r l}\textbf{Режим 1}  & ТЕЛЕКОММУНИКАЦИИ И СВЯЗИ ВОЗДЕЙСТВИЕ \\ 
\textbf{Режим 2}  & НА ПАРОЛЬНО-КЛЮЧЕВЫЕ СИСТЕМЫ ЗАЩИТЫ \\ 
\textbf{Режим 2}  & Индикаторная буква: Я \\ 
& АВТОМАТИЗИРОВАННЫХ СИСТЕМ ОБРАБОТКИ И \\ 
\textbf{Режим 4}  & ПЕРЕДАЧИ ИНФОРМАЦИИ КОМПРОМЕТАЦИЯ КЛЮЧЕЙ \\ 
\textbf{Режим 5}  & Пароль: КРИПТОГРАФИЧЕСКОЙ \\ 
& И СРЕДСТВ КРИПТОГРАФИЧЕСКОЙ ЗАЩИТЫ \\ 
	\end{tabular} 
\end{table}

\end{solution}
\begin{exercise}\begin{table}[H]
	\centering
	\begin{tabular}{r l}\textbf{Режим 1}  & ИАРЯАНЙБШЮ ЬНМСПЛ ЕПОЦПДСЮЫБ ГВ \\ 
\textbf{Режим 2}  & ЮЛШДЛЮАДБАБ АЗТЮЬЧИ ЧНЖПНЦОПЕ БПГМШЧЙЕДКЧ \\ 
\textbf{Режим 3}  & Индикаторная буква: М \\ 
& АЬЬЭХЧККВ ЩЙТ ПЛЦБЖСНЕФ АЫЪБЫЯЖВКХ \\ 
\textbf{Режим 4}  & ЬяцужълшМжсхзжсьЫгщзорыжКфщкфръуАщ \\ 
\textbf{Режим 5}  & Пароль: ИНФОРМАЦИИ \\ 
& ЧДФБДЭЩЙ Ы ДТДСЭРЫЩ ХГПЪДЛОИЗЩ \\ 
	\end{tabular} 
\end{table}

\end{exercise}
\begin{solution}
\begin{table}[H]
	\centering
	\begin{tabular}{r l}\textbf{Режим 1}  & ИНФОРМАЦИИ УТЕЧКА ИНФОРМАЦИИ ПО \\ 
\textbf{Режим 2}  & ТЕХНИЧЕСКИМ КАНАЛАМ ВНЕДРЕНИЕ ЭЛЕКТРОННЫХ \\ 
\textbf{Режим 2}  & Индикаторная буква: М \\ 
& УСТРОЙСТВ ДЛЯ ПЕРЕХВАТА ИНФОРМАЦИИ \\ 
\textbf{Режим 4}  & В ТЕХНИЧЕСКИЕ СРЕДСТВА ОБРАБОТКИ \\ 
\textbf{Режим 5}  & Пароль: ИНФОРМАЦИИ \\ 
& ХРАНЕНИЯ И ПЕРЕДАЧИ ИНФОРМАЦИИ \\ 
	\end{tabular} 
\end{table}

\end{solution}
\begin{exercise}\begin{table}[H]
	\centering
	\begin{tabular}{r l}\textbf{Режим 1}  & ПЙ ГСЪИРЭЛ ГФЗИИ Ю \\ 
\textbf{Режим 2}  & ЮАГЫУ С ЕЛЕСИХУЦК ТЯИШВЖФЮГ \\ 
\textbf{Режим 3}  & Индикаторная буква: Щ \\ 
& ОХГЛЧЧЫ ЫЙПЧТИДДДЭИБУУИ ШИПААЗ ННЦМХХЙЫОДД \\ 
\textbf{Режим 4}  & АюхплрылЪхжюжкэмЦцихоцлхГжойпгдчНебвг \\ 
\textbf{Режим 5}  & Пароль: УНИЧТОЖЕНИЕ \\ 
& ПМ ПЯЧУМ ЬГЬЬШЬТСЙЫБСБ ЦФЪЖЙЯВНЯША \\ 
	\end{tabular} 
\end{table}

\end{exercise}
\begin{solution}
\begin{table}[H]
	\centering
	\begin{tabular}{r l}\textbf{Режим 1}  & ПО КАНАЛАМ СВЯЗИ А \\ 
\textbf{Режим 2}  & ТАКЖЕ В СЛУЖЕБНЫЕ ПОМЕЩЕНИЯ \\ 
\textbf{Режим 2}  & Индикаторная буква: Щ \\ 
& ОРГАНОВ ГОСУДАРСТВЕННОЙ ВЛАСТИ ПРЕДПРИЯТИЙ \\ 
\textbf{Режим 4}  & УЧРЕЖДЕНИЙ И ОРГАНИЗАЦИЙ НЕЗАВИСИМО \\ 
\textbf{Режим 5}  & Пароль: УНИЧТОЖЕНИЕ \\ 
& ОТ ФОРМЫ СОБСТВЕННОСТИ УНИЧТОЖЕНИЕ \\ 
	\end{tabular} 
\end{table}

\end{solution}
\begin{exercise}\begin{table}[H]
	\centering
	\begin{tabular}{r l}\textbf{Режим 1}  & ПЙТЙУУЯЯРАИ РЗНБЛГЗИОЖ ШОЖ СПАДЖЕЩ \\ 
\textbf{Режим 2}  & ФЕБЖОЗУН М ЩЮЗЫЩШ ДДМБЕСБИЯ \\ 
\textbf{Режим 3}  & Индикаторная буква: Р \\ 
& ЙЖКЧЖМАЗРЛ ДАЪСМФОВ ИАРЯАНЙБШЮ Е \\ 
\textbf{Режим 4}  & ЙяюамбьМжьжчмыШикшттцЙбо \\ 
\textbf{Режим 5}  & Пароль: ДЕШИФРОВАНИЕ \\ 
& КЕ ЯЫРЙЗИ ЧЬНЮЮ ПФШТЭРЗУНГШП \\ 
	\end{tabular} 
\end{table}

\end{exercise}
\begin{solution}
\begin{table}[H]
	\centering
	\begin{tabular}{r l}\textbf{Режим 1}  & ПОВРЕЖДЕНИЕ РАЗРУШЕНИЕ ИЛИ ХИЩЕНИЕ \\ 
\textbf{Режим 2}  & МАШИННЫХ И ДРУГИХ НОСИТЕЛЕЙ \\ 
\textbf{Режим 2}  & Индикаторная буква: Р \\ 
& ИНФОРМАЦИИ ПЕРЕХВАТ ИНФОРМАЦИИ В \\ 
\textbf{Режим 4}  & СЕТЯХ ПЕРЕДАЧИ ДАННЫХ И \\ 
\textbf{Режим 5}  & Пароль: ДЕШИФРОВАНИЕ \\ 
& НА ЛИНИЯХ СВЯЗИ ДЕШИФРОВАНИЕ \\ 
	\end{tabular} 
\end{table}

\end{solution}
\begin{exercise}\begin{table}[H]
	\centering
	\begin{tabular}{r l}\textbf{Режим 1}  & ЭЙГБ ТЛДЫЛАЮЩФК Х НМЙЪЧЫЪВЮМГ \\ 
\textbf{Режим 2}  & ЫВФЪХУУЧЮЭФРЯ Ч ЦВОКЛЩЩЙНШ РЪПГЪЛКМСБТАМР \\ 
\textbf{Режим 3}  & Индикаторная буква: Ь \\ 
& ЦТПВВБЛФЬЬ ЫУОШААЦ ЮДБЧЭУ ЙЖКЧЖМАЗРЛ \\ 
\textbf{Режим 4}  & МсьжчсяОксбъырЦнцтхоцАощщжлдЦшфьннуЙнокйыоЦхх \\ 
\textbf{Режим 5}  & Пароль: СОЗДАНИИ \\ 
& ЦКШЖЩ ЪЕЫ ЦЫИЦАЩЫЫ Р \\ 
	\end{tabular} 
\end{table}

\end{exercise}
\begin{solution}
\begin{table}[H]
	\centering
	\begin{tabular}{r l}\textbf{Режим 1}  & ЭТОЙ ИНФОРМАЦИИ И НАВЯЗЫВАНИЕ \\ 
\textbf{Режим 2}  & ОТЕЧЕСТВЕННЫХ И ЗАРУБЕЖНЫХ ИНФОРМАЦИОННЫХ \\ 
\textbf{Режим 2}  & Индикаторная буква: Ь \\ 
& ТЕХНОЛОГИЙ СРЕДСТВ ЗАЩИТЫ ИНФОРМАЦИИ \\ 
\textbf{Режим 4}  & СРЕДСТВ ИНФОРМАТИЗАЦИИ ТЕЛЕКОММУНИКАЦИИ И \\ 
\textbf{Режим 5}  & Пароль: СОЗДАНИИ \\ 
& СВЯЗИ ПРИ СОЗДАНИИ И \\ 
	\end{tabular} 
\end{table}

\end{solution}
\begin{exercise}\begin{table}[H]
	\centering
	\begin{tabular}{r l}\textbf{Режим 1}  & РЗНЗЭБКХ СНЬИЖЖФЧШЙ ЕПОЦПДСЮЫГВРРА СУЩБЖУУЯТХССЭЫ \\ 
\textbf{Режим 2}  & ЩЯЪКЫАДИТЯФАЬПМФКЛЖ ЕЗХХХЖ Е ГМЮЕМЩНПТА \\ 
\textbf{Режим 3}  & Индикаторная буква: Г \\ 
& ОЩИШГБИВЪЧС А СИЩЫФМ Я \\ 
\textbf{Режим 4}  & ЭнэтэцсМмээщуэНндйюащПкзхпэсБппск \\ 
\textbf{Режим 5}  & Пароль: РАСПРОСТРАНЕНИЕ \\ 
& ШБЙПЮСЗРЮЩЫ ФН ЕЕХЦЙСХЩЙТЮЛЩША ЫФЭЙЙИРЕРГ. \\ 
	\end{tabular} 
\end{table}

\end{exercise}
\begin{solution}
\begin{table}[H]
	\centering
	\begin{tabular}{r l}\textbf{Режим 1}  & РАЗВИТИИ РОССИЙСКОЙ ИНФОРМАЦИОННОЙ ИНФРАСТРУКТУРЫ \\ 
\textbf{Режим 2}  & НЕСАНКЦИОНИРОВАННЫЙ ДОСТУП К ИНФОРМАЦИИ \\ 
\textbf{Режим 2}  & Индикаторная буква: Г \\ 
& НАХОДЯЩЕЙСЯ В БАНКАХ И \\ 
\textbf{Режим 4}  & БАЗАХ ДАННЫХ НАРУШЕНИЕ ЗАКОННЫХ \\ 
\textbf{Режим 5}  & Пароль: РАСПРОСТРАНЕНИЕ \\ 
& ОГРАНИЧЕНИЙ НА РАСПРОСТРАНЕНИЕ ИНФОРМАЦИИ. \\ 
	\end{tabular} 
\end{table}

\end{solution}
\begin{exercise}\begin{table}[H]
	\centering
	\begin{tabular}{r l}\textbf{Режим 1}  & ИЫЫЯВИОХШ СМФОЙ ЙЖКЧЖМАЗРДГЩЩУ ФИЬББЮБССААЗ \\ 
\textbf{Режим 2}  & ОЮЗШЪЪЧГЕР ПАССЩХЕЬИ УАЗАЦЗЧЧОГВЭКЛ ЖЫ \\ 
\textbf{Режим 3}  & Индикаторная буква: Э \\ 
& ИРЬЦБЬФ Н ШСБТСЖФКЖЙ. Ж \\ 
\textbf{Режим 4}  & НьщающбЧьфубящЙнокйэиМяэнсеяФъкьскдЫгцачжзМяг \\ 
\textbf{Режим 5}  & Пароль: ВОЕННЫХ \\ 
& ЦБЭЕМШХШБВП ЪЖУЦЩШЖАЗТЦД ГЛГЧЛЮСБАЕЙЯИ КЭАЩЧЖИ \\ 
	\end{tabular} 
\end{table}

\end{exercise}
\begin{solution}
\begin{table}[H]
	\centering
	\begin{tabular}{r l}\textbf{Режим 1}  & ИСТОЧНИКИ УГРОЗ ИНФОРМАЦИОННОЙ БЕЗОПАСНОСТИ \\ 
\textbf{Режим 2}  & РОССИЙСКОЙ ФЕДЕРАЦИИ ПОДРАЗДЕЛЯЮТСЯ НА \\ 
\textbf{Режим 2}  & Индикаторная буква: Э \\ 
& ВНЕШНИЕ И ВНУТРЕННИЕ. К \\ 
\textbf{Режим 4}  & ВНЕШНИМ ИСТОЧНИКАМ ОТНОСЯТСЯ ДЕЯТЕЛЬНОСТЬ \\ 
\textbf{Режим 5}  & Пароль: ВОЕННЫХ \\ 
& ИНОСТРАННЫХ ПОЛИТИЧЕСКИХ ЭКОНОМИЧЕСКИХ ВОЕННЫХ \\ 
	\end{tabular} 
\end{table}

\end{solution}
\begin{exercise}\begin{table}[H]
	\centering
	\begin{tabular}{r l}\textbf{Режим 1}  & РЗНЗХЗДЧЙЯЧОРОБИ Й МЧХПЧЕЯШЛЪЩВШМ ЩЩЛВЬЫЫА \\ 
\textbf{Режим 2}  & ЩЭЩЩХКАФАЯЧЦ ССНСЮЕ ЧЗФДЗГШПБ ПМЦЙЫЫДБРА \\ 
\textbf{Режим 3}  & Индикаторная буква: Л \\ 
& ФГВВЧАЗРЛ С БВМРВУЗЪНАСЬЬХ САПЭЙ \\ 
\textbf{Режим 4}  & ЙяасюШаяфтСрбйьВдвкфЧтчсн \\ 
\textbf{Режим 5}  & Пароль: ИНТЕРЕСОВ \\ 
& АГМШЮШЙЫГИЩГЪ Й ЯХНЮВАЙШЫ ШДЩЕЕАЖЭЩ \\ 
	\end{tabular} 
\end{table}

\end{exercise}
\begin{solution}
\begin{table}[H]
	\centering
	\begin{tabular}{r l}\textbf{Режим 1}  & РАЗВЕДЫВАТЕЛЬНЫХ И ИНФОРМАЦИОННЫХ СТРУКТУР \\ 
\textbf{Режим 2}  & НАПРАВЛЕННАЯ ПРОТИВ ИНТЕРЕСОВ РОССИЙСКОЙ \\ 
\textbf{Режим 2}  & Индикаторная буква: Л \\ 
& ФЕДЕРАЦИИ В ИНФОРМАЦИОННОЙ СФЕРЕ \\ 
\textbf{Режим 4}  & СТРЕМЛЕНИЕ РЯДА СТРАН К \\ 
\textbf{Режим 5}  & Пароль: ИНТЕРЕСОВ \\ 
& ДОМИНИРОВАНИЮ И УЩЕМЛЕНИЮ ИНТЕРЕСОВ \\ 
	\end{tabular} 
\end{table}

\end{solution}
\begin{exercise}\begin{table}[H]
	\centering
	\begin{tabular}{r l}\textbf{Режим 1}  & РБЛЫЭФ Ж НОЬЭМКЗ ПШЭЮШЦУХЩМДЪЪВ \\ 
\textbf{Режим 2}  & ЯЯИССНЪОФФЩЪ РСШРЧЕБЪЫФ ЯС Г \\ 
\textbf{Режим 3}  & Индикаторная буква: Н \\ 
& ЬВСПЫНКУ Н ШСБТСЖФКЦЕО ЦЮФЖЮЙТОВЕЪГОЦ \\ 
\textbf{Режим 4}  & ЗлпуябШоузулЛхзъжмМжфжючЛкжлзюГмохжюГхоюифЛъжнмм \\ 
\textbf{Режим 5}  & Пароль: ТЕХНОЛОГИЯМИ \\ 
& ЩЕ ЛИТЛЭГГВЖ ЫЙАЛДУЛЕЧЪЕЭХМШ ГАДЖЫШЪУЗНМШ \\ 
	\end{tabular} 
\end{table}

\end{exercise}
\begin{solution}
\begin{table}[H]
	\centering
	\begin{tabular}{r l}\textbf{Режим 1}  & РОССИИ В МИРОВОМ ИНФОРМАЦИОННОМ \\ 
\textbf{Режим 2}  & ПРОСТРАНСТВЕ ВЫТЕСНЕНИЮ ЕЕ С \\ 
\textbf{Режим 2}  & Индикаторная буква: Н \\ 
& ВНЕШНЕГО И ВНУТРЕННЕГО ИНФОРМАЦИОННЫХ \\ 
\textbf{Режим 4}  & РЫНКОВ ОБОСТРЕНИЕ МЕЖДУНАРОДНОЙ КОНКУРЕНЦИИ \\ 
\textbf{Режим 5}  & Пароль: ТЕХНОЛОГИЯМИ \\ 
& ЗА ОБЛАДАНИЕ ИНФОРМАЦИОННЫМИ ТЕХНОЛОГИЯМИ \\ 
	\end{tabular} 
\end{table}

\end{solution}
\begin{exercise}\begin{table}[H]
	\centering
	\begin{tabular}{r l}\textbf{Режим 1}  & И БХУБЯЯМКЮ ЙЙВФДПАППЖЖН ЪУУЯМРКВЫФТЙВ \\ 
\textbf{Режим 2}  & ЫВЫЙШЧ ЖШЖАЛЮЯ ММХЕЛР ЦЪЧА \\ 
\textbf{Режим 3}  & Индикаторная буква: В \\ 
& Ц ЮЛШБНГУЯДЛТ БЦ НРСУБЭААУУШХ \\ 
\textbf{Режим 4}  & ВозохзфцДмфцпшиьБлдтшчъвСсдрищедЭымъявещБшжчшчыпШцг \\ 
\textbf{Режим 5}  & Пароль: ДЕЯТЕЛЬНОСТЬ \\ 
& НЭПЖШЙЙЛСЮ ГЪЯЭЦМЕНЛГБДИЖ ЬЧШЙЭПДССВ ЛУЪСЫЕЩЖДЕЩЭ \\ 
	\end{tabular} 
\end{table}

\end{exercise}
\begin{solution}
\begin{table}[H]
	\centering
	\begin{tabular}{r l}\textbf{Режим 1}  & И РЕСУРСАМИ ДЕЯТЕЛЬНОСТЬ МЕЖДУНАРОДНЫХ \\ 
\textbf{Режим 2}  & ОТРЫВА ВЕДУЩИХ ДЕРЖАВ МИРА \\ 
\textbf{Режим 2}  & Индикаторная буква: В \\ 
& И НАРАЩИВАНИЕ ИХ ВОЗМОЖНОСТЕЙ \\ 
\textbf{Режим 4}  & ПО ПРОТИВОДЕЙСТВИЮ СОЗДАНИЮ КОНКУРЕНТОСПОСОБНЫХ \\ 
\textbf{Режим 5}  & Пароль: ДЕЯТЕЛЬНОСТЬ \\ 
& РОССИЙСКИХ ИНФОРМАЦИОННЫХ ТЕХНОЛОГИЙ ДЕЯТЕЛЬНОСТЬ \\ 
	\end{tabular} 
\end{table}

\end{solution}
\begin{exercise}\begin{table}[H]
	\centering
	\begin{tabular}{r l}\textbf{Режим 1}  & КНЬХЖАМКПМФ ЛПЩЫЖОЪХП РВЩЩАЬГ Н \\ 
\textbf{Режим 2}  & ЩЭАЬББПГ ЛХРЬХЛЖЬЗЖС П ЦЮАФ \\ 
\textbf{Режим 3}  & Индикаторная буква: Т \\ 
& ЖПЫУЙЙЬ (ИТИЛК) ЛШЭШОШФО ЗФФННКВЮЖЯК \\ 
\textbf{Режим 4}  & ВъзкэдвЦяатчяцЯрцяамчЧнчтпзяВюъзкэдЛлзохв \\ 
\textbf{Режим 5}  & Пароль: ПРЕДУСМАТРИВАЮЩИХ \\ 
& ЭШФТВМБЩЪ ЙГНЙБШИМКШФКЬЮ ЦЙЪЮ ДХЛХНМБПКИЮЫСДФГН \\ 
	\end{tabular} 
\end{table}

\end{exercise}
\begin{solution}
\begin{table}[H]
	\centering
	\begin{tabular}{r l}\textbf{Режим 1}  & КОСМИЧЕСКИХ ВОЗДУШНЫХ МОРСКИХ И \\ 
\textbf{Режим 2}  & НАЗЕМНЫХ ТЕХНИЧЕСКИХ И ИНЫХ \\ 
\textbf{Режим 2}  & Индикаторная буква: Т \\ 
& СРЕДСТВ (ВИДОВ) РАЗВЕДКИ ИНОСТРАННЫХ \\ 
\textbf{Режим 4}  & ГОСУДАРСТВ РАЗРАБОТКА РЯДОМ ГОСУДАРСТВ \\ 
\textbf{Режим 5}  & Пароль: ПРЕДУСМАТРИВАЮЩИХ \\ 
& КОНЦЕПЦИЙ ИНФОРМАЦИОННЫХ ВОЙН ПРЕДУСМАТРИВАЮЩИХ \\ 
	\end{tabular} 
\end{table}

\end{solution}
\begin{exercise}\begin{table}[H]
	\centering
	\begin{tabular}{r l}\textbf{Режим 1}  & СЭЖЙГЗЦД ОШЩРЖЖТ ЕЕИЕВВНЛ КБНХХЭТТПЗД \\ 
\textbf{Режим 2}  & ЩЭ ТЛДЫЛАЮЩФСИНЩЧ ИСЙКУ ЪЗХРЪО \\ 
\textbf{Режим 3}  & Индикаторная буква: П \\ 
& ТТСМФ ХЖФЩ ШРЗХНВЙЩЫ ДДЕВЭЫШЫЫКУ \\ 
\textbf{Режим 4}  & ЭдудеДпхифГсфгоЙнияаСррджФвглфХмялыОбчдрЕчьяиДифм \\ 
\textbf{Режим 5}  & Пароль: ДОСТУПА \\ 
& ЭЫГШФЗМЮН ДРЮПМЗЕРЦЦКПФКСЙЦСБФ ЭЕЖШБЦД А \\ 
	\end{tabular} 
\end{table}

\end{exercise}
\begin{solution}
\begin{table}[H]
	\centering
	\begin{tabular}{r l}\textbf{Режим 1}  & СОЗДАНИЕ СРЕДСТВ ОПАСНОГО ВОЗДЕЙСТВИЯ \\ 
\textbf{Режим 2}  & НА ИНФОРМАЦИОННЫЕ СФЕРЫ ДРУГИХ \\ 
\textbf{Режим 2}  & Индикаторная буква: П \\ 
& СТРАН МИРА НАРУШЕНИЕ НОРМАЛЬНОГО \\ 
\textbf{Режим 4}  & СИСТЕМ СОХРАННОСТИ ИНФОРМАЦИОННЫХ РЕСУРСОВ \\ 
\textbf{Режим 5}  & Пароль: ДОСТУПА \\ 
& ПОЛУЧЕНИЕ НЕСАНКЦИОНИРОВАННОГО ДОСТУПА К \\ 
	\end{tabular} 
\end{table}

\end{solution}
\begin{exercise}\begin{table}[H]
	\centering
	\begin{tabular}{r l}\textbf{Режим 1}  & НЗК. З ЕОНФОЪЧПЪМ ЩППДЮЩУБКБ \\ 
\textbf{Режим 2}  & ЫВТТБПУАМ ХИЮИЧЯЕХЦЖЩ ЮЦЧЧЕИГБЯ ВЪИМЭЛЛШОЬИГУ \\ 
\textbf{Режим 3}  & Индикаторная буква: Ц \\ 
& ЖИСЯЦЪФФФЩЪЧПЬО Щ ГЙЛЩБВФБЮБПГ ЫЫТБФААИ \\ 
\textbf{Режим 4}  & ХфяыдКрщыкЮгнтаЙнихяЧлотоВозжэЖгхяэЦй \\ 
\textbf{Режим 5}  & Пароль: ДОСТУПА \\ 
& ЗРРМЪЬАОЖДИЙК ЧЬРЧДШБПДУР ЛПТЖСЛС Д \\ 
	\end{tabular} 
\end{table}

\end{exercise}
\begin{solution}
\begin{table}[H]
	\centering
	\begin{tabular}{r l}\textbf{Режим 1}  & НИМ. К ВНУТРЕННИМ ИСТОЧНИКАМ \\ 
\textbf{Режим 2}  & ОТНОСЯТСЯ КРИТИЧЕСКОЕ СОСТОЯНИЕ ОТЕЧЕСТВЕННЫХ \\ 
\textbf{Режим 2}  & Индикаторная буква: Ц \\ 
& ГОСУДАРСТВЕННЫХ И КРИМИНАЛЬНЫХ СТРУКТУР \\ 
\textbf{Режим 4}  & В ИНФОРМАЦИОННОЙ СФЕРЕ ПОЛУЧЕНИЯ \\ 
\textbf{Режим 5}  & Пароль: ДОСТУПА \\ 
& КРИМИНАЛЬНЫМИ СТРУКТУРАМИ ДОСТУПА К \\ 
	\end{tabular} 
\end{table}

\end{solution}
\begin{exercise}\begin{table}[H]
	\centering
	\begin{tabular}{r l}\textbf{Режим 1}  & КНФЯЖММШИМЫЙСЙЙЩ РЪПГЪЛКМСЬ ВЫЭИЗИОЕ ЙОЖЩЗЦЛ \\ 
\textbf{Режим 2}  & ЫЛХЮЯФХЬПМФККЖ ХХДОООЧЙЙППЛ ГН АТЯЫЖ \\ 
\textbf{Режим 3}  & Индикаторная буква: У \\ 
& ПУФЫЦЦСН ДРАЬНУИЖ ББЗСЮЭШ ЖЪУПАДЖЧЧЮЮЩ \\ 
\textbf{Режим 4}  & РмрпшГооысЮналкВхекзОкфроЬбтьъКрфчьИмйси \\ 
\textbf{Режим 5}  & Пароль: ИНФОРМАЦИОННОЙ \\ 
& Ы СВВФЧАЯМЩЬН К ОГЦВРЖМОХЪБЩГЯ \\ 
	\end{tabular} 
\end{table}

\end{exercise}
\begin{solution}
\begin{table}[H]
	\centering
	\begin{tabular}{r l}\textbf{Режим 1}  & КОНФИДЕНЦИАЛЬНОЙ ИНФОРМАЦИИ УСИЛЕНИЯ ВЛИЯНИЯ \\ 
\textbf{Режим 2}  & ОРГАНИЗОВАННОЙ ПРЕСТУПНОСТИ НА ЖИЗНЬ \\ 
\textbf{Режим 2}  & Индикаторная буква: У \\ 
& ОБЩЕСТВА СНИЖЕНИЯ СТЕПЕНИ ЗАЩИЩЕННОСТИ \\ 
\textbf{Режим 4}  & ЗАКОННЫХ ИНТЕРЕСОВ ГРАЖДАН ОБЩЕСТВА \\ 
\textbf{Режим 5}  & Пароль: ИНФОРМАЦИОННОЙ \\ 
& И ГОСУДАРСТВА В ИНФОРМАЦИОННОЙ \\ 
	\end{tabular} 
\end{table}

\end{solution}
\begin{exercise}\begin{table}[H]
	\centering
	\begin{tabular}{r l}\textbf{Режим 1}  & САПЭЙ ЗЕЪЮЗЗУЮЙОДЬН ЛЩВЩНСУШГЭП ШШМЯЧОРООККМ \\ 
\textbf{Режим 2}  & МЬННЛШСЙСДЫ ИЬЦДХХЪ ЪШОХВУЧЧЧАУГЩЩУ ЭУОРРН \\ 
\textbf{Режим 3}  & Индикаторная буква: Й \\ 
& ИЬЦДХХЪ ЪШОХВУЧЧЧАУГЩЩУ ЭУОРРН БЫЧГЮХСФЕ \\ 
\textbf{Режим 4}  & РошххйцЮыьтивкМчжьмбзШибущуыУйъжпыуЛжмб \\ 
\textbf{Режим 5}  & Пароль: ГОСУДАРСТВЕННОЙ \\ 
& Ч РБУОЩМСХЦШ АЯСНХЯ ТПИЮЪСЧКСЬАБНХЯ \\ 
	\end{tabular} 
\end{table}

\end{exercise}
\begin{solution}
\begin{table}[H]
	\centering
	\begin{tabular}{r l}\textbf{Режим 1}  & СФЕРЕ НЕДОСТАТОЧНАЯ КООРДИНАЦИЯ ДЕЯТЕЛЬНОСТИ \\ 
\textbf{Режим 2}  & ФЕДЕРАЛЬНЫХ ОРГАНОВ ГОСУДАРСТВЕННОЙ ВЛАСТИ \\ 
\textbf{Режим 2}  & Индикаторная буква: Й \\ 
& ОРГАНОВ ГОСУДАРСТВЕННОЙ ВЛАСТИ СУБЪЕКТОВ \\ 
\textbf{Режим 4}  & РОССИЙСКОЙ ФЕДЕРАЦИИ ПО ФОРМИРОВАНИЮ \\ 
\textbf{Режим 5}  & Пароль: ГОСУДАРСТВЕННОЙ \\ 
& И РЕАЛИЗАЦИИ ЕДИНОЙ ГОСУДАРСТВЕННОЙ \\ 
	\end{tabular} 
\end{table}

\end{solution}
\begin{exercise}\begin{table}[H]
	\centering
	\begin{tabular}{r l}\textbf{Режим 1}  & ПЙЪРЦЫЛУ Н РКТЗЛЛЭ ЯЧШЯИЧУЦОПР \\ 
\textbf{Режим 2}  & ЫЩЙВЩБОДИТЯССК ЦЖЮФФГФШШООД ЖЦЧПРРМЛЩУ ЕИЭЭЫЮЩФК \\ 
\textbf{Режим 3}  & Индикаторная буква: П \\ 
& ЧЦЗЩ ИПМИЮЦОКУИВЪ ЙЧДДШАВТХ Ф \\ 
\textbf{Режим 4}  & ВцшнзхКыкйарУццпдюРэдодрВфвчме \\ 
\textbf{Режим 5}  & Пароль: ОБЩЕСТВА \\ 
& СПФСРЮЭЖЫЛ ДИСВДАБШЖЭТЦ ЗТЦЩФСАТ Ц \\ 
	\end{tabular} 
\end{table}

\end{exercise}
\begin{solution}
\begin{table}[H]
	\centering
	\begin{tabular}{r l}\textbf{Режим 1}  & ПОЛИТИКИ В ОБЛАСТИ ОБЕСПЕЧЕНИЯ \\ 
\textbf{Режим 2}  & ИНФОРМАЦИОННОЙ БЕЗОПАСНОСТИ РОССИЙСКОЙ ФЕДЕРАЦИИ \\ 
\textbf{Режим 2}  & Индикаторная буква: П \\ 
& БАЗЫ РЕГУЛИРУЮЩЕЙ ОТНОШЕНИЯ В \\ 
\textbf{Режим 4}  & ИНФОРМАЦИОННОЙ СФЕРЕ А ТАКЖЕ \\ 
\textbf{Режим 5}  & Пароль: ОБЩЕСТВА \\ 
& ИНСТИТУТОВ ГРАЖДАНСКОГО ОБЩЕСТВА И \\ 
	\end{tabular} 
\end{table}

\end{solution}
\begin{exercise}\begin{table}[H]
	\centering
	\begin{tabular}{r l}\textbf{Режим 1}  & НЖПФННЩФШИЧСД ЫЙПЧТИДДДЭИБУЦИ ИЯЬТСНХВ ЧГ \\ 
\textbf{Режим 2}  & ОЛДЛЪЮЩЫЪ ЛГЦЩГЫХЕЬУАЯЯЮЬ СЩФЗЪ ФОКХМЕ \\ 
\textbf{Режим 3}  & Индикаторная буква: К \\ 
& ЛИЭБЛЛЖБСРННЧ ЯЖОЩШХМШЧЫТМВБ ГТДЩЩЩЯЗВИИ ТЯ \\ 
\textbf{Режим 4}  & ЩимемкНнчббщЪцйкниЩщэоязЗзшжябФрвлтпЦшафцйЮятец \\ 
\textbf{Режим 5}  & Пароль: СИСТЕМЫ \\ 
& ТЪЦШПМШЦЙАТ ЬЭЕЫТДГЧ ЪКЖТЩЮШЦЧЕМПГ ЬЗЗПТЪИ \\ 
	\end{tabular} 
\end{table}

\end{exercise}
\begin{solution}
\begin{table}[H]
	\centering
	\begin{tabular}{r l}\textbf{Режим 1}  & НЕДОСТАТОЧНЫЙ ГОСУДАРСТВЕННЫЙ КОНТРОЛЬ ЗА \\ 
\textbf{Режим 2}  & РАЗВИТИЕМ ИНФОРМАЦИОННОГО РЫНКА РОССИИ \\ 
\textbf{Режим 2}  & Индикаторная буква: К \\ 
& НЕДОСТАТОЧНОЕ ФИНАНСИРОВАНИЕ МЕРОПРИЯТИЙ ПО \\ 
\textbf{Режим 4}  & ФЕДЕРАЦИИ НЕДОСТАТОЧНАЯ ЭКОНОМИЧЕСКАЯ МОЩЬ \\ 
\textbf{Режим 5}  & Пароль: СИСТЕМЫ \\ 
& ГОСУДАРСТВА СНИЖЕНИЕ ЭФФЕКТИВНОСТИ СИСТЕМЫ \\ 
	\end{tabular} 
\end{table}

\end{solution}
\begin{exercise}\begin{table}[H]
	\centering
	\begin{tabular}{r l}\textbf{Режим 1}  & ОЩОЛДЧЫТМВЬ Л СГДЩАГОУИЖ ЯОШЬЯЯЕЬКАЗЗД \\ 
\textbf{Режим 2}  & УЛТЬДКББЖЬ КЦЕОЖЬПЬЙЗЖЛУЙМЭЮ ВИЯЪВЭ Ф \\ 
\textbf{Режим 3}  & Индикаторная буква: Ы \\ 
& ЖЫМТЧЧР ДЬАДЩЬЙНУИЖ ЭСЫЬСФЕТЖХЗШШЙ ЛЩГЦЦЯЦЪЪЕЕТ \\ 
\textbf{Режим 4}  & ЖяхоаЖбыжыЦьрицРлрпкЙопниЛохтфЖхохуЧчимсЮйщтбШхштуЗх \\ 
\textbf{Режим 5}  & Пароль: ГОСУДАРСТВЕННОЙ \\ 
& ЕЫЦЗЦАОЦЮГЧЩЩЫХ ДТСЮЬЩ ШЙЫВФГЬ ФИЭГУУНЧКАРШФГЫ \\ 
	\end{tabular} 
\end{table}

\end{exercise}
\begin{solution}
\begin{table}[H]
	\centering
	\begin{tabular}{r l}\textbf{Режим 1}  & ОБРАЗОВАНИЯ И ВОСПИТАНИЯ НЕДОСТАТОЧНОЕ \\ 
\textbf{Режим 2}  & КОЛИЧЕСТВО КВАЛИФИЦИРОВАННЫХ КАДРОВ В \\ 
\textbf{Режим 2}  & Индикаторная буква: Ы \\ 
& ОБЛАСТИ ОБЕСПЕЧЕНИЯ ИНФОРМАЦИОННОЙ БЕЗОПАСНОСТИ \\ 
\textbf{Режим 4}  & НЕДОСТАТОЧНАЯ АКТИВНОСТЬ ФЕДЕРАЛЬНЫХ ОРГАНОВ \\ 
\textbf{Режим 5}  & Пароль: ГОСУДАРСТВЕННОЙ \\ 
& ГОСУДАРСТВЕННОЙ ВЛАСТИ ОРГАНОВ ГОСУДАРСТВЕННОЙ \\ 
	\end{tabular} 
\end{table}

\end{solution}
\begin{exercise}\begin{table}[H]
	\centering
	\begin{tabular}{r l}\textbf{Режим 1}  & ВЦБЮЮЩ ПЧСОАЫДРФ РБЛЫЭЭТФИО АПЙЙКВИМЕ \\ 
\textbf{Режим 2}  & Г ЙЖКЧЖМЩЦЕЯИЩУТ РКЦЭЛЛШЧ С \\ 
\textbf{Режим 3}  & Индикаторная буква: Э \\ 
& ДНРИЯ ФФЖЫОНДННЬЬЖ Д КВМЯБШЦРМВР \\ 
\textbf{Режим 4}  & ТпчгйгКыкьыюЧртогоТйгъбзЪкэожэВзгвшцФт \\ 
\textbf{Режим 5}  & Пароль: РЕСУРСОВ \\ 
& ШСЩЖТПЧИ ГЭЦЗЪСРККЙБЦЮФД ХДЬЧЖХЕК Ц \\ 
	\end{tabular} 
\end{table}

\end{exercise}
\begin{solution}
\begin{table}[H]
	\centering
	\begin{tabular}{r l}\textbf{Режим 1}  & ВЛАСТИ СУБЪЕКТОВ РОССИЙСКОЙ ФЕДЕРАЦИИ \\ 
\textbf{Режим 2}  & В ИНФОРМИРОВАНИИ ОБЩЕСТВА О \\ 
\textbf{Режим 2}  & Индикаторная буква: Э \\ 
& СВОЕЙ ДЕЯТЕЛЬНОСТИ В РАЗЪЯСНЕНИИ \\ 
\textbf{Режим 4}  & ПРИНИМАЕМЫХ РЕШЕНИЙ В ФОРМИРОВАНИИ \\ 
\textbf{Режим 5}  & Пароль: РЕСУРСОВ \\ 
& ОТКРЫТЫХ ГОСУДАРСТВЕННЫХ РЕСУРСОВ И \\ 
	\end{tabular} 
\end{table}

\end{solution}
\begin{exercise}\begin{table}[H]
	\centering
	\begin{tabular}{r l}\textbf{Режим 1}  & РЗНЗЭБКХ ЯЗЬЬЦЗБ ДШОООЧТ Ъ \\ 
\textbf{Режим 2}  & ЩУЛ НВОНКЮЯ ЯУААМЙДХЧМ ХШОЗДЪ \\ 
\textbf{Режим 3}  & Индикаторная буква: Х \\ 
& ЛЩ ХФХЛЪКЫ ЯЯИДХ ЗПХР \\ 
\textbf{Режим 4}  & ЕнэяиэДмкгюкЙуисэйЧбфкчвЧффлозЯыцяънЩзбь \\ 
\textbf{Режим 5}  & Пароль: ОРГАНОВ \\ 
& ЫОЕАЩЫГ ФШЭЮЯОХВКДЛЩБЗЯ ЩШАЕЩЦ ЫОЕАЩЫГ \\ 
	\end{tabular} 
\end{table}

\end{exercise}
\begin{solution}
\begin{table}[H]
	\centering
	\begin{tabular}{r l}\textbf{Режим 1}  & РАЗВИТИИ СИСТЕМЫ ДОСТУПА К \\ 
\textbf{Режим 2}  & НИМ ГРАЖДАН ОТСТАВАНИЕ РОССИИ \\ 
\textbf{Режим 2}  & Индикаторная буква: Х \\ 
& ОТ ВЕДУЩИХ СТРАН МИРА \\ 
\textbf{Режим 4}  & ПО УРОВНЮ ИНФОРМАТИЗАЦИИ ФЕДЕРАЛЬНЫХ \\ 
\textbf{Режим 5}  & Пароль: ОРГАНОВ \\ 
& ОРГАНОВ ГОСУДАРСТВЕННОЙ ВЛАСТИ ОРГАНОВ \\ 
	\end{tabular} 
\end{table}

\end{solution}
\begin{exercise}\begin{table}[H]
	\centering
	\begin{tabular}{r l}\textbf{Режим 1}  & ГЮЗОРТПППЯТЩВВТ ФТЗЛЛЭ УБПЩЖОЬКД КЗХОЕЕЮДЙЩ \\ 
\textbf{Режим 2}  & МЬННЛШГЭФ К ЬСПЕККД ШЕХХЧЧЛЦ \\ 
\textbf{Режим 3}  & Индикаторная буква: Г \\ 
& ЭМЖУОЙЧЛЦ ЗМЛИЛЕЕНК ЛХЛШЭАЮПИОЕ ЮЦЭГЪЗШЭШБЦРМВЬ \\ 
\textbf{Режим 4}  & НецадСфчцгФеитэИхйи \\ 
\textbf{Режим 5}  & Пароль: ПОСЛЕДНИЕ \\ 
& ЮРСДПДЩ. БЕ ЯЫЦПЧЦЩЫЧ ЮЫУА \\ 
	\end{tabular} 
\end{table}

\end{exercise}
\begin{solution}
\begin{table}[H]
	\centering
	\begin{tabular}{r l}\textbf{Режим 1}  & ГОСУДАРСТВЕННОЙ ВЛАСТИ СУБЪЕКТОВ РОССИЙСКОЙ \\ 
\textbf{Режим 2}  & ФЕДЕРАЦИИ И ОРГАНОВ МЕСТНОГО \\ 
\textbf{Режим 2}  & Индикаторная буква: Г \\ 
& СЕЛЬСКОГО ХОЗЯЙСТВА ОБРАЗОВАНИЯ ЗДРАВООХРАНЕНИЯ \\ 
\textbf{Режим 4}  & СФЕРЫ УСЛУГ И БЫТА \\ 
\textbf{Режим 5}  & Пароль: ПОСЛЕДНИЕ \\ 
& ГРАЖДАН. ЗА ПОСЛЕДНИЕ ГОДЫ \\ 
	\end{tabular} 
\end{table}

\end{solution}
\end{document}