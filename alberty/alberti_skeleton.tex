\documentclass[a4paper,14pt]{extarticle}
\usepackage[utf8]{inputenc}
\usepackage[T2A]{fontenc}
\usepackage{titlesec}
\usepackage{tabu}
\usepackage{booktabs}% for better rules in the table
\usepackage{lipsum}
\usepackage{amsmath}
\usepackage{amsthm}
\usepackage{lscape}
\usepackage{amsfonts}
\usepackage{mathrsfs}
\usepackage{multicol}
\usepackage[russian]{babel}
\usepackage{graphicx}
\usepackage[landscape]{geometry}
\usepackage{pgfplots}
\usepackage{float}
\geometry{left=2.5cm}
\geometry{right=2.5cm}
\geometry{top=2.5cm}
\geometry{bottom=2.5cm}
\pgfplotsset{compat=1.9}
\usepackage{alphalph}

%
%
%\parindent=0.75cm
%
\frenchspacing
%\sloppy
\hyphenpenalty=50

\usepackage{array}
\usepackage{makecell}
\newcolumntype{x}[1]{>{\centering\arraybackslash}p{#1}}

\usepackage{tikz}
\usetikzlibrary{calc}
\usepackage{zref-savepos}

\newcounter{DiagonalizedEntry}
\renewcommand*{\theDiagonalizedEntry}{NTE-\the\value{DiagonalizedEntry}}

\newcommand*{\diagonalize}[2]{%
  \multicolumn{1}{@{}c@{}|}{%
    \stepcounter{DiagonalizedEntry}%
    \vadjust pre{\zsavepos{\theDiagonalizedEntry t}}% top
    \vadjust{\zsavepos{\theDiagonalizedEntry b}}% bottom
    \zsavepos{\theDiagonalizedEntry l}% left
    \hspace{0pt plus 1filll}%
    \zsavepos{\theDiagonalizedEntry r}% right
    \tikz[overlay]{%
      \draw[black]
        let
          \n{llx}={\zposx{\theDiagonalizedEntry l}sp-\zposx{\theDiagonalizedEntry r}sp}, % x left
          \n{urx}={0}, % x right
          \n{lly}={\zposy{\theDiagonalizedEntry b}sp-\zposy{\theDiagonalizedEntry r}sp}, % y bottom
          \n{ury}={\zposy{\theDiagonalizedEntry t}sp-\zposy{\theDiagonalizedEntry r}sp} %  y top
    in
        (\n{llx}, \n{ury}) -- (\n{urx}, \n{lly})
    node[anchor=south west] at (\n{llx}, \n{lly}) {#1}
    node[anchor=north east] at (\n{urx}, \n{ury}) {#2}
    ;
    }% 
  }%
}

%
%---------------------------------------------------------------
\newtheorem{theorem}{Теорема}
\newtheorem{definition}[theorem]{Определение}
\newtheorem{remark}[theorem]{Замечание}
\newtheorem{lemma}[theorem]{Лемма}
\newtheorem{utv}[theorem]{Утверждение}
\newtheorem{proposition}[theorem]{Предложение}
\newtheorem{example}[theorem]{Пример}
\newtheorem{corollary}[theorem]{Следствие}
\newtheorem{prop}[theorem]{Утверждение}
%

\newcommand*{\expect}{\mathsf{M}}
\newcommand*{\prob}{\mathsf{P}}
\newcommand{\eps}{\varepsilon}

%\renewcommand{\tg}{\mbox{tg}\,}
%\renewcommand{\ctg}{\mbox{ctg}\,}
%\renewcommand{\arctg}{\mbox{arctg}\,}
%\renewcommand{\arcctg}{\mbox{arcctg}\,}
%\renewcommand{\arcsin}{\mbox{arcsin}\,}
%\renewcommand{\arccos}{\mbox{arccos}\,}
%\renewcommand{\sh}{\mbox{sh}\,}
%\renewcommand{\ch}{\mbox{ch}\,}
\title{Семинар по математическому анализу} 
\author{А.Б. Чухно} 
\newcommand{\thedate}{\today}
\date{\thedate} 

\newcounter{exercise}[section]
\newenvironment{exercise}[1][]{\refstepcounter{exercise}\par\medskip
%\noindent\makebox[\linewidth]{\rule{\textwidth}{2.25pt}}+
   \noindent\textbf{Вариант~\theexercise. #1}\\
   \noindent\makebox[\linewidth]{\rule{\textwidth}{1.25pt}}
   }
{\vspace{-2.5px}\mbox{}\newline \noindent\makebox[\linewidth]{\rule{\textwidth}{.5pt}}
}

\usepackage{verbatim}

\newenvironment{solution}
{\begin{proof}[\textbf{\textit{Решение}}]}
  {\end{proof}}
%\let\solution\comment
%\let\endsolution\endcomment
\def\ansrel{\mathrel{\stackrel{\mbox{почему?}}{=} }}

\begin{document}
\pagestyle{empty}


\begin{center}
    \textbf{Диск Альберти}
\end{center}
\begin{tikzpicture}


\node[circle,thick,draw=black,fill=white,minimum size=14cm]() at (-2,-8) {};

\node[circle,thick,draw=black,fill=white,minimum size=10cm]() at (11,-8) {};
\foreach \j in {1,...,6}
{
    \pgfmathsetmacro{\jx}{-2+7*cos(11.25*\j) };
    \pgfmathsetmacro{\jy}{-8+7*sin(11.25*\j) };
    \pgfmathsetmacro{\jxx}{-2+6*cos(11.25*\j -5.625) };
    \pgfmathsetmacro{\jyy}{-8+6*sin(11.25*\j-5.625) };
    \pgfmathtruncatemacro{\jsy}{\j+223};
    
    \draw[-,thick](-2, -8) -- (\jx,\jy);
    \node[]() at (\jxx,\jyy) {\symbol{\jsy}};
}
\foreach \j in {7,...,32}
{
    \pgfmathsetmacro{\jx}{-2+7*cos(11.25*\j) };
    \pgfmathsetmacro{\jy}{-8+7*sin(11.25*\j) };
    \pgfmathsetmacro{\jxx}{-2+6*cos(11.25*\j -5.625) };
    \pgfmathsetmacro{\jyy}{-8+6*sin(11.25*\j-5.625) };
    \pgfmathtruncatemacro{\jsy}{\j+223};
    
    \draw[-,thick](-2, -8) -- (\jx,\jy);
    \node[]() at (\jxx,\jyy) {\symbol{\jsy}};
}



\foreach \j in {1,...,6}
{
    \pgfmathsetmacro{\jx}{11+5*cos(11.25*\j) };
    \pgfmathsetmacro{\jy}{-8+5*sin(11.25*\j) };
    
    
    \draw[-,thick](11, -8) -- (\jx,\jy);
    
}
\foreach \j in {7,...,33}
{
    \pgfmathsetmacro{\jx}{11+5*cos(11.25*\j) };
    \pgfmathsetmacro{\jy}{-8+5*sin(11.25*\j) };
    
    
    \draw[-,thick](11, -8) -- (\jx,\jy);
    
}
\end{tikzpicture}

\newpage

\begin{exercise}


\textbf{\textit{Ключ к расшифрованияю:} К\,\, Ф \,\,Э\,\, Н\,\, И\,\, Ь\,\, С\,\, Я\,\, А\,\, Т\,\, У\,\, Б\,\, Ы\,\, Л\,\, Р\,\, В\,\, Щ\,\, Г\,\, Ъ\,\, Д\,\, Е\,\, М\,\, Й\,\, Ц\,\, П\,\, Ч\,\, Ж\,\, Ю\,\, Ш\,\, З\,\, О\,\, Х\,\,
}


\begin{table}[H]
    \centering
    \begin{tabular}{r l}
      \textbf{Режим 1}   & ТГСДУЖ ШМФ ПЮШЧМГ, ЪМЙДВАИШ А РЫАСТШЯШЛГ
 \\
         \textbf{Режим 2}   & ФВ Ч КЕЮВХМП ШПДПЭПДТ ТЯЫКЫЖЭ.
 \\
         \textbf{Режим 3}   & Индикаторная буква: Ы \\

& ВВВ ЛЮХР ДЬДЯДКЗЩ БДНДОЬДЮ \\
         \textbf{Режим 4}   &  ШиояшлГнчнйшЯсдвжйЛзлвмхДпоъхдЩзщзщаУвъйт\\
         \textbf{Режим 5}   & Пароль: Стеганография \\
         & ЦПЭЮЙАБ, ДДЧСВ, С, ЯИЙГДГ, ЭЙЧЙЛ \\
    \end{tabular}

\end{table}


\end{exercise}\begin{solution}

\begin{table}[H]
    \centering
    \begin{tabular}{r l}
      \textbf{Режим 1}   & ТЫСЯЧИ ЛЕТ КОРОЛИ, КОРОЛЕВЫ И ПОЛКОВОДЦЫ

 \\
         \textbf{Режим 2}   & МЫ Ж ОГЛАСИМ СОКРЫТОЕ ЖЕЛАНЬЕ.

 \\
         \textbf{Режим 3}   & Индикаторная буква: Ы \\

& ГДЕ ДУШИ ОБРЕТАЮТ ОЧИЩЕНЬЕ\\
         \textbf{Режим 4}   &  и властителям, и управителям. А на наше имя\\
         \textbf{Режим 5}   & Пароль: Стеганография \\
         & СПОКОЕН, ВЕСЕЛ, Я, БЫВАЛО, ДЕЛОМ \\
    \end{tabular}
\end{table}





\end{solution}


\end{document}